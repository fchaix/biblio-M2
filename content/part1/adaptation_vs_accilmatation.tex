Avant de rentrer plus profondément dans le sujet, il me semble nécessaire de
préciser une subtilité de vocabulaire utile lorsque l'on discute d'une réponse
à un stress. Il s'agit de faire une différence entre la notion d'acclimatation
et celle d'adaptation évolutive, souvent confondues dans la langage courant.
La différence se fait essentiellement sur l'échelle de temps durant laquelle
se déroule le phénomène.

Dans le cas de l'adaptation, il s'agit d'une échelle de temps relativement
longue, durant laquelle les organismes seront confrontés à un changement de
condition environementale, induisant une pression de séléction différente des
conditions initiales. De cela va découler une séléction, et une adaptation de
la lignée par modification séléctive des gènes les plus adaptés. L'adaptation
est donc une réponse stable et héritable à un stress.

Dans le cas d'une acclimatation, nous sommes dans le cas d'une échelle de
temps beaucoup plus réduite. Les organismes se retrouvent confrontés à une
situation stressante, et va, plus ou moins efficacement, exprimer un phénotype
d'adaptation, que l'on peut assimiler à une resistance à cet environement
stressant. Il s'agit d'une réponse graduelle et souvent réversible, opérée à
l'échelle de l'individu.

% \begin{note}
% 	Commenter l'hypothèse qu'une adaptation serait une suite logique à l'acclimatation, lorsque le stress dure longtemps ?
% \end{note}

% Paragraphe sur ce que l'on dit, nous

Dans ce mémoire, nous allons nous intéresser plus en détail aux phénomènes
d'acclimatation suite à un stress ponctuel, exercé à l'échelle des individus.
Cependant, il ne faut pas oublier que l'adaptation génomique à un stress à
long terme peut être une piste également pertinente pour l'étude de la
dynamique des espèces invasives.

Pour reprendre le cas de la résistance au froid, évoqué en introduction, nous
pourrions évoquer la piste des molécules anti-gel. L'utilisation de ces
molécules se retrouvant à la fois dans le groupe des insectes\cite{duman2001}
que dans celui des bactéries\cite{xu1998}, nous pouvons aisément supposer des
intéractions, d'autant plus que certaines de ces molécules sont extrêmement
simples et ubiquistes (petite peptides, petits sucres).
