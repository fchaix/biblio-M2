\cite{federhoffmann1999, zhang2011}

Le terme HSP (\eng{Heat Shock Proteins}) désigne une grande famille de
protéines initialement décrites comme étant induite spécifiquement par un
stress thermique chaud. Par la suite, elles se sont avérées jouer un rôle bien
plus large dans la réponse générique aux stress de diverses
natures\cite{sorensen2003}\footnote{Tableau Sørensen pour illustrer}.

% Rôles

Ces protéines font pour la plupart partie du grand groupe des protéines chaperonnes. Les
protéines chaperonnes jouent des rôles multiples tournent tous autour de la
gestion de la conformation des protéines, de leur repliement, leur appariement
dans des complexes protéiques, ou encore le transport.  Leur rôle est crucial
dans le processus de formation de protéines, y compris dans un environement
non stressant. Cependant, lors d'un stress, le besoin de réparation rapide des
structures endomagées par ce stress se fait plus fort, d'où l'existance de
cette sous-famille de chaperonnes que sont les HSPs.

% Fort taux de concervation

Les HSPs ont un taux de concervation dans le vivant extrêmement important, ce
qui fait que leurs gènes sont souvent califiées de gènes dits <<~de ménage~>>
(\eng{“Housekeeping genes”} en anglais)

% Familles

Ces HSPs sont classiquement classées dans des grandes familles en fonction de
leur poids moléculaire \cite{fink1999} (par exemple, les protéines de la
famille de Hsp70 ont un poids moléculaire d'environ 70\,kDa), mais de récentes
tentatives de clarification de cette nomenlature tendraient vers des noms ne
faisant pas allusion au poids moléculaire\footnote{Kampinga et al., 2009. À
voir si je le cite ou pas, juste pour ça...}

\subsubsection{Les petites HSPs} % (fold)
\label{ssub:les_petites_hsps}

  Les petites HSPs, dont le poids moléculaire se situe entre 12 et 43\,kDa,
  contrairement aux autres familles ne sont exprimmées que lors d'un choc
  thermique. Leur rôle est assez mal connu\footnote{Mettre à jour, peut-
  être...}, mais elles sont décrites comme se liant aux protéines dénaturées.
  Le rôle de cette liaison peut être celui d'empêcher les protéines dénaturées
  de s'aggréger, ce qui gênerait l'action des HSPs chaperonnes.

\subsubsection{La famille HSP40} % (fold)
\label{ssub:la_famille_hsp40}

  La famille HSP40, aussi appelée DnaJ, contient des protéines hautement
  concervées, carractérisées par la présence du domaine J, qui correspond au
  domaine hautement concervé de la protéine. Le rôle le plus décrit des
  protéines de ce groupe est celui de cochaperonnes pour HSP70. D'autres rôles
  sont mis en évidence dans la littérature, comme un rôle dans l'adressage à
  l'ubiquitine \cite{lee1996}.

\subsubsection{La famile HSP60} % (fold)
\label{ssub:la_famile_hsp60}

  La famille HSP60, aussi appelées chaperonines (cpn60), comprend deux grands
  complexes protéiques : GroEL et TCP-1. GroEL et ses homologues se retrouvent
  dans les organismes procaryotes (dont la mitochondrie et le chloroplaste),
  alors que TCP-1 est exprimmé dans le cytoplasme de la cellule eucaryote.

  Dans la bactérie, la présence de la co-chaperonine GroES (couplée à l'ATP)
  est nécessaire pour le fonctionnement de GroEL.

  Certains membres de cette famille entrent aussi dans la constitution de la
  Ribulose-1,5-bisphosphate carboxylase oxygenase (RuBisCO), protéine centrale
  du processus de photosynthèse chez les plantes.

\subsubsection{La famille HSP70} % (fold)
\label{ssub:la_famille_hsp70}

  Les protéines de la famille HSP70 sont très nombreuses, et ont souvent
  plusieurs représentants dans un seul organisme (par exemple, la plupart des
  eucaryotes en ont plus de 10 différents, répartis dans tous les
  compartiments cellulaires). Elles jouent un rôle de chaperones, associées
  aux co-chaperonnes DnaJ et GrpE. Nous en retiendrons une qui semble plus que
  les autres induite par le stress, BiP (ou Grp78), une HSP70 cytoplasmique.

\subsubsection{La famille HSP90} % (fold)
\label{ssub:la_famille_hsp90}

\todo[inline]{Citation : \citet{chiosis2013} ssi j’arrive à obtenir un accès (Nature Structural \& Molecular Biology).\\
Les publis que j’ai trouvé jusqu’à maintenant semblent dire que les rôles in vivo de HSP90 sont mal connus, mais elles (les publis) sont un peut vieilles. Dans celle-là, peut-être que je trouverais plus de choses.}

  Les HSP90 sont, à l'instar de HSP40, des protéines hautement concervées. On
  en retrouve dans tous les organismes, qu’ils soient eucaryotes ou
  procaryotes. Elles sont souvent associées à HP70, dans leur rôle de
  chaperonnes. On leur attribue aussi des rôles dans les mécanismes de
  transduction du signal, et en association avec le cytosquelette.

\subsubsection{La famille HSP100} % (fold)
\label{ssub:la_famille_hsp100}


%\input{content/figures/tab_HSP}
