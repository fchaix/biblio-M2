\documentclass[%
	a4paper,
	10pt,
	linktocpage=true,
	oneside,
	twocolumn,
	DIV=calc, 
	% DIV règle les proportions corps de texte-marges. Classic, c'est pour
	% suivre les canons du Moyen-âge. Pardonnez-moi, prince.
	]{scrreprt}
% scrartcl est la classe qui remplace la classe « article » de LaTeX. En gros
% c'est la même chose, mais avec des choses mieux pour la typographie des
% langues européennes, la gestion du nombre de caractères par ligne, etc.

\usepackage{shapepar} % Ça c’est pour le joli calligramme à la fin

\documentclass[%
	a4paper,
	10pt,
	linktocpage=true,
	oneside,
	twocolumn,
	DIV=calc, 
	% DIV règle les proportions corps de texte-marges. Classic, c’est pour
	% suivre les canons du Moyen-âge. Pardonnez-moi, prince.
	]{scrartcl}
% scrartcl est la classe qui remplace la classe « article » de LaTeX. En gros
% c’est la même chose, mais avec des choses mieux pour la typographie des
% langues européennes, la gestion du nombre de caractères par ligne, etc.

\usepackage[francais]{babel}
\usepackage{fontspec}
    \setmainfont[%
	Numbers=OldStyle,
	Ligatures={
		TeX,
		Common,
		Rare,
		Historical
		}
	]{Linux Libertine O}
    % \setmonofont{Ubuntu Mono}
    \setsansfont{Linux Biolinum O}

\KOMAoptions{DIV=last} 
% La définition de DIV, dans le cas de calc, doit se déclarer après la
% déclaration de la police.

\usepackage[%
    hidelinks=true
    colorlinks=false
    pdfauthor={François Chaix},
    pdftitle={Document de travail — recherches biblio François Chaix},
    pdfdisplaydoctitle=true, % Display document title instead of filename in title bar
    pdfsubject={plop},
    pdfkeywords={plop},
    pdfproducer={LuaTeX, avec le package hyperref},
    pdfcreator={LuaTex},
    linktocpage=false,
    pdfinfo={pouet ?},
    pdflang={fr-FR},
    unicode=true,
    verbose=true
    ]{hyperref}

\usepackage{lipsum}

% Cet environnement Figure ne servira que si on est obligé d’insérer une image
% dans un environnement multicol. Mais c’est moche.
\newenvironment{Figure}
  {\par\medskip\noindent\minipage{\linewidth}}
  {\endminipage\par\medskip}

%%%%%%%% Concernant la biblio %%%%%%%%%%%
\usepackage[
%	style=aem,
	natbib=true,
	backend=biber
	]{biblatex}
\addbibresource{bib.bib}

%%%%%%%% Concernant le document %%%%%%%%%
\title{Mon super titre que je trouverais plus tard}
\author{François \textsc{Chaix}}

\begin{document}
\nocite{*}

\twocolumn[
\begin{@twocolumnfalse}
	\maketitle
	\begin{abstract}

Ceci est un document de travail pour la rédaction de mon mémoire
bibliographique de recherche pour mon M2. La version twocolumns n’est
évidement pas la version finale (les gens fixant les règles de présentation
ayant des goûts douteux en terme de typographie), mais la version dédiée au
travail sur le fond (ça fait moins de pages à imprimer, et y’a pas à dire,
c’est plus lisible.). Merci de votre compréhension.

\end{abstract}
	\tableofcontents
  \vspace{1cm}
\end{@twocolumnfalse}
]

	\section{Introduction} % (fold)
\label{sec:introduction}

\chapter{Réponses de l'insecte au stress de température} % (fold)
\label{chap:repstress}
	
	\section{Généralités} % (fold)
	\label{sec:g_n_ralit_s}
		\input{content/part1/généralités}

		\subsection{Adaptation vs acclimatation} % (fold)
		\label{sub:adaptation_vs_accilmatation}
			Avant de rentrer plus profondément dans le sujet, il est nécessaire de
rappeler une confusion souvent commise lorsque l'on discute d'une réponse à un
stress. Il s'agit de faire une différence entre la notion d'acclimatation et
celle d'adaptation, souvent confondues dans la langage courant. La différence
se fait essentiellement sur l'échelle de temps durant laquelle se déroule le
phénomène.

Dans le cas de l'adaptation, il s'agit d'une échelle de temps relativement
long, durant laquelle les organismes seront confrontés à un changement de
condition environementale, induisant une pression de séléction différente des
conditions initiales. De cela va découler une séléction, et une adaptation de
la lignée par modification séléctive des gènes les plus adaptés. L'adaptation
est donc une modification stable et héritable à un stress.

Dans le cas d'une acclimatation, nous sommes dans le cas d'une échelle de
temps beaucoup plus réduite. Les organismes se retrouvent confrontés à une
situation stressante, et va, plus ou moins efficacement, exprimer un phénotype
d'adaptation, que l'on peut assimiler à une resistance à cet environement
stressant. Il s'agit d'une réponse graduelle et souvent réversible, opérée à
l'échelle de l'individu.

\begin{note}
	Commenter l'hypothèse qu'une adaptation serait une suite logique à l'acclimatation, lorsque le stress dure longtemps ?
\end{note}

% Paragraphe sur ce que l'on dit, nous

Dans ce mémoire, nous allons nous intéresser plus en détail aux phénomènes
d'acclimatation suite à un stress ponctuel, exercé à l'échelle des individus.
Cependant, il ne faut pas oublier que l'adaptation génomique à un stress à
long terme peut être une piste également pertinente pour l'étude de la
dynamique des espèces invasives.

Pour reprendre le cas de la résistance au froid, évoqué en introduction, nous
pourrions évoquer la piste des molécules anti-gel. L'utilisation de ces
molécules se retrouvant à la fois dans le groupe des insectes\cite{duman2001} que dans celui
des bactéries\cite{xu1998}, nous pouvons aisément supposer des intéractions, d'autant plus
que certaines de ces molécules sont extrêmement simples et ubiquistes (petite
peptides, petits sucres).


%		\subsection{Quelques exemples d'adaptation aux températures extrêmes} % (fold)
%		\label{sub:exemples_adaptations}
%			\begin{note}

C'est pas certain que l'on garde cette partie, un peut HS.

Ici c’est pour parler de protéines anti-gel, de trucs anti-dessication pour les temp chaudes\ldots

\end{note} % Huhu, EndNote ! lol…

		\subsection{Les Heat-Shock Proteins (HSP), clé de voûte de la réponse acclimative} % (fold)
		\label{sub:generalites_HSP}
			\cite{federhoffmann1999, zhang2011}

Le terme HSP (\eng{Heat Shock Proteins}) désigne une grande famille de
protéines initialement décrites comme étant induite spécifiquement par un
stress thermique chaud. Par la suite, elles se sont avérées jouer un rôle bien
plus large dans la réponse générique aux stress de diverses
natures\cite{sorensen2003}\footnote{Tableau Sørensen pour illustrer}.

% Rôles

Ces protéines font pour la plupart partie du grand groupe des protéines chaperonnes. Les
protéines chaperonnes jouent des rôles multiples tournent tous autour de la
gestion de la conformation des protéines, de leur repliement, leur appariement
dans des complexes protéiques, ou encore le transport.  Leur rôle est crucial
dans le processus de formation de protéines, y compris dans un environement
non stressant. Cependant, lors d'un stress, le besoin de réparation rapide des
structures endomagées par ce stress se fait plus fort, d'où l'existance de
cette sous-famille de chaperonnes que sont les HSPs.

% Fort taux de concervation

Les HSPs ont un taux de concervation dans le vivant extrêmement important, ce
qui fait que leurs gènes sont souvent califiées de gènes dits <<~de ménage~>>
(\eng{“Housekeeping genes”} en anglais)

% Familles

Ces HSPs sont classiquement classées dans des grandes familles en fonction de
leur poids moléculaire \cite{fink1999} (par exemple, les protéines de la
famille de Hsp70 ont un poids moléculaire d'environ 70\,kDa), mais de récentes
tentatives de clarification de cette nomenlature tendraient vers des noms ne
faisant pas allusion au poids moléculaire\footnote{Kampinga et al., 2009. À
voir si je le cite ou pas, juste pour ça...}

\subsubsection{Les petites HSPs} % (fold)
\label{ssub:les_petites_hsps}

  Les petites HSPs, dont le poids moléculaire se situe entre 12 et 43\,kDa,
  contrairement aux autres familles ne sont exprimmées que lors d'un choc
  thermique. Leur rôle est assez mal connu\footnote{Mettre à jour, peut-
  être...}, mais elles sont décrites comme se liant aux protéines dénaturées.
  Le rôle de cette liaison peut être celui d'empêcher les protéines dénaturées
  de s'aggréger, ce qui gênerait l'action des HSPs chaperonnes.

\subsubsection{La famille HSP40} % (fold)
\label{ssub:la_famille_hsp40}

  La famille HSP40, aussi appelée DnaJ, contient des protéines hautement
  concervées, carractérisées par la présence du domaine J, qui correspond au
  domaine hautement concervé de la protéine. Le rôle le plus décrit des
  protéines de ce groupe est celui de cochaperonnes pour HSP70. D'autres rôles
  sont mis en évidence dans la littérature, comme un rôle dans l'adressage à
  l'ubiquitine \cite{lee1996}.

\subsubsection{La famile HSP60} % (fold)
\label{ssub:la_famile_hsp60}

  La famille HSP60, aussi appelées chaperonines (cpn60), comprend deux grands
  complexes protéiques : GroEL et TCP-1. GroEL et ses homologues se retrouvent
  dans les organismes procaryotes (dont la mitochondrie et le chloroplaste),
  alors que TCP-1 est exprimmé dans le cytoplasme de la cellule eucaryote.

  Dans la bactérie, la présence de la co-chaperonine GroES (couplée à l'ATP)
  est nécessaire pour le fonctionnement de GroEL.

  Certains membres de cette famille entrent aussi dans la constitution de la
  Ribulose-1,5-bisphosphate carboxylase oxygenase (RuBisCO), protéine centrale
  du processus de photosynthèse chez les plantes.

\subsubsection{La famille HSP70} % (fold)
\label{ssub:la_famille_hsp70}

  Les protéines de la famille HSP70 sont très nombreuses, et ont souvent
  plusieurs représentants dans un seul organisme (par exemple, la plupart des
  eucaryotes en ont plus de 10 différents, répartis dans tous les
  compartiments cellulaires). Elles jouent un rôle de chaperones, associées
  aux co-chaperonnes DnaJ et GrpE. Nous en retiendrons une qui semble plus que
  les autres induite par le stress, BiP (ou Grp78), une HSP70 cytoplasmique.

\subsubsection{La famille HSP90} % (fold)
\label{ssub:la_famille_hsp90}

\todo[inline]{Citation : \citet{chiosis2013} ssi j’arrive à obtenir un accès (Nature Structural \& Molecular Biology).\\
Les publis que j’ai trouvé jusqu’à maintenant semblent dire que les rôles in vivo de HSP90 sont mal connus, mais elles (les publis) sont un peut vieilles. Dans celle-là, peut-être que je trouverais plus de choses.}

  Les HSP90 sont, à l'instar de HSP40, des protéines hautement concervées. On
  en retrouve dans tous les organismes, qu’ils soient eucaryotes ou
  procaryotes. Elles sont souvent associées à HP70, dans leur rôle de
  chaperonnes. On leur attribue aussi des rôles dans les mécanismes de
  transduction du signal, et en association avec le cytosquelette.

\subsubsection{La famille HSP100} % (fold)
\label{ssub:la_famille_hsp100}


%\input{content/figures/tab_HSP}


	\section{Gènes impliqués dans la réponse aux stress thermiques, et régulations} % (fold)
	\label{sec:genes}
		Outre les gènes codant pour les HSPs, 

% Il y a aussi (Armstrong et al), avec cette histoire de K+ dans le cerveau…
	% section genes (end)

	\section{Exemples chez les modèles d'insectes les plus étudiés} % (fold)
	\label{sec:exemples_modeles}
		Nous allons maintenant détailler quelques exemples de cas dans lesquels le
partenaire microbien a un rôle prépondérent dans les mécanismes de réaction au
stress thermique de leur hôte. 


		\subsection{Exemple droso} % (fold)
		\label{sub:exemple_droso}
			\input{content/part1/exemple_droso}

		\subsection{Exemple moustique} % (fold)
		\label{sub:exemple_moustique}
			\input{content/part1/exemple_moustique}

		\subsection{Exemple puceron} % (fold)
		\label{sub:exemple_puceron}
			\input{content/part1/exemple_puceron}

% chapter stressinsect (end)
\chapter{Implication du microbiote dans la réponse au stress de température} % (fold)
\label{sec:implicationµbiote}
	
	\section{Exemples clés} % (fold)
	\label{sec:exemples}
		Nous allons maintenant détailler quelques exemples de cas dans lesquels le
partenaire microbien a un rôle prépondérent dans les mécanismes de réaction au
stress thermique de leur hôte. 

	% section exemples (end)

		\subsection{\esp{Rickettsia sp.} et les aleurodes (\textit{whiteflies})} % (fold)
		\label{sub:rickettsia_et_les_aleurodes_}
			\paragraph{description} % (fold)
\label{par:description_whitefly}

\esp{Rickettsia sp.} est un symbionte secondaire de esp{Bemisia tabaci}, un
diptère causant des maladies au tabac. Une étude \cite{brumin2011} a récement
mis en évidence un mécanisme conduisant à une meilleure résistance de la
mouche au stress thermique, lorsqu'elle est associée à sa bactérie
symbiotique.

\paragraph{Processus mis en jeu} % (fold)
\label{par:process_whitefly}

La présence de la bactérie dans les tissus de l'hôte induit une expression de
gènes associés au stress, de façon permanente, même dans des conditions
thermiques favorables. Cette présence continue de protéines de stress joue un
rôle préventif : Lors d'un stress thermique, l'insecte aura déjà exprimmé ses
gènes de résistance et résistera donc mieux au froid.

		% subsection rickettsia_et_les_aleurodes_ (end)
	
% chapter implicationµbiote (end)
\chapter*{Perspectives}
\addcontentsline{toc}{chapter}{Perspectives}

\todo[inline]{Limite entre acclimatation et adatation pas sur la même échelle de temps entre le microbiote et la bestiole, ce qui fait que quand l'insecte ne peut pas s'adapter (de façon héritable), le microbiote, ayant fait plusieurs générations, a eu le temps d'ajuster son génome par séléction (entre autres).}

\todo[inline]{Pas sûr de mettre cette partie, j'ai déjà trop de pages, et la rédaction de certaines parties n'est pas encore terminée...}

	
\printbibliography

%\diamondpar{
%	
%}

\Canflagshape{
	Si vous venez de committer quelque chose mais que vous vous rendez compte que vous devez réparer ce commit, les versions récentes de git commit vous donnent accès à l’option --amend qui demande à git de remplacer le commit de HEAD par un autre, basé sur le contenu actuel de l’index. Cela vous donne l’opportunité d’ajouter de fichiers que vous avez oubliés ou de corriger des erreurs de typo dans le message du commit, avant de publier les changements pour les autre développeurs.	
}

\end{document}
