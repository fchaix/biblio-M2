%Exemple : partie \ref{sub:groel}

\subsection{Processus mis en jeu} % (fold)
\label{sub:processus_mis_en_jeu2}

Le cas de l'expression d'une protéine de choc thermique GroEL par les
organelles issues d'une ancienne endosymbiose (voir partie
\ref{ssub:la_famile_hsp60}) très similaire à celle produite par les bactéries
suggère la possibilité d'une utilisation de l'expression bactérienne de
protéines de réponse au stress, par leur hôte. On observe effectivement ce
type de mécanisme assez souvent, nous en donnerons quelques exemples.

\subsection{Exemples} % (fold)
\label{sub:exemples_genes_bactos}

% Exemple Bushnera et mutation gène ibpA (dunbar2007)

Notre premier exemple portera sur l'association entre \esp{Bushnera sp.} et le
puceron \esp{Acyrthosiphon pisum}. Une étude \cite{dunbar2007} a montré qu'une
simple mutation dans le promoteur d'une petite HSP de \esp{Bushnera} induisait
chez l'hôte du mutant une résistance moindre aux choc thermiques.
\todo[inline]{À expliquer : C'est une mutation assez commune}

% Exemple GroEL

Dans notre deuxième exemple, nous utiliserons le cas évoqué dans
\ref{sub:processus_mis_en_jeu2} de la protéine GroEL sur-exprimée par
l'endosymbionte \esp{Blochmannia floridanus} en cas de stress, et qui,
relarguée dans l'hémolymphe de son hôte, la fourmi \esp{Campontus floridanus}.
Une fois dans l'hémolymphe, les protéines de stress bactériennes peuvent
exercer leurs fonctions de chaperonne sur les protéines dénaturées de l'hôte
(\cite{stoll2009}). \todo[inline]{Peut-être trouver un autre exemple, celui-ci n'est
pas clairement le propos de la publi}
