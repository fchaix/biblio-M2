\begin{table}
\begin{center}
\begin{tabular}{ l c | r }
	\multicolumn{2}{c|}{Protéines} & Fonction \\
	Nomenclature classique & Autres noms & \\
	\hline
	\hline
	\multicolumn{2}{c|}{Famille HSP70} & \\
	\hline
	Hsp70 & HSPA1 & plop \\
	Grp75 & HSPA9 & plop \\
	Grp78 & HSPA5, BiP & plop \\
	\hline
	\multicolumn{2}{c}{Famille HSP110}& \\
	\hline
	Hsp110 & HSPH2 & plop \\
	\hline
	\multicolumn{2}{c}{Famille HSP60}& \\
	\hline
	Hsp60 & HSPD1, GroEL & plop \\
	\hline
	\multicolumn{2}{c}{Famille HSP40}& \\
	\hline
	Hsp40 & DNAJB1 & plop \\
	\hline
	\multicolumn{2}{c}{Famille sHSP}& \\
	\hline
	\alpha{}A & HSPB4 & plop \\
	\alpha{}B & HSPB5 & plop \\
\end{tabular}
\caption{Inventaire des HSPs et de leurs multiples nominations.(\citet{zhang2011}). Pas fini, pas sûr que ce soit utile.}
\label{tab:tab1}
\end{center}
\end{table}
