\chapter{Réponses de l'insecte aux stress de température}

% Préambule accilmatation vs. adaptation
En biologie tout stress appelle une réponse de l'organisme pour maintenir l'équilibre biologique dans un état fonctionnel.
Deux types de réponses, souvent confondues à tort, sont observées lors d'un stress : l'adaptation et l'acclimatation.
La différence se situe essentiellement sur l'échelle de temps durant laquelle se déroule le phénomène.
Dans le cas de l'adaptation, les organismes seront confrontés à un changement environnemental sur l'échelle de plusieurs générations.
Ce changement sur le long terme va introduire une pression de séléction qui va à plus ou moins long terme modifier le génome de la lignée (ou la composition de l'holobionte), qui sera plus adaptée à ce nouvel environnement.
Das le cas de l'acclimatation, l'échelle de temps est réduite à celle de la vie d'un individu. La réponse au changement environnemental sera dépendant des capacités intrinsèques de l'individu ou de l'holobionte en question, et ces changements seront réversibles.
On parle aussi de plasticité adaptative de l'individu pour désigner cette capacité d'acclimatation.
Ce mémoire sera plus particulièrement consacré à l'étude des
phénomènes d'acclimatation suite à un stress ponctuel dans le temps, exercé à l'échelle des individus.

	\section{Les protéines de choc thermique, clé de voûte de la réponse acclimative}

	La famille de protéines de réponse au stress que nous allons évoquer dans cette partie ne concerne pas uniquement les insectes.
	Les HSP (\eng{Heat Shock Proteins}) constituent en effet la réponse la plus universelle au stress thermique qu'il est possible de trouver, des bactéries aux vertébrés supérieurs.
	Elles sont par conséquent de loin les plus documentées dans la littérature.
	Leur découverte remonte à 1974 où Tissière \textit{et al.} les mit en évidence par éléctrophorèse dans des extraits tissulaires de drosophiles ayant subi un stress chaud \cite{tissieres1974}.
	Par la suite, elles se sont avérées jouer un rôle bien plus large dans la réponse générique aux stress de diverses natures, comme les radiations aux UV, l'hypoxie et d'autres stress, listés dans le tableau 1 \cite{sorensen2003}.

	Ces protéines appartiennent pour la plupart au grand groupe des protéines chaperonnes \cite{federhoffmann1999}.
	Celles-ce jouent des rôles multiples, en particulier celui de la gestion de la conformation des protéines, leur appariement dans des complexes protéiques, ou encore dans le transport.
	Leur rôle est crucial dans le processus de formation de protéines, y compris dans des conditions environementales non stressantes.
	Mais quand survient un stress le besoin accru de réparation rapide des structures endomagées est nécessaire, c'est donc en réponse à ce besoin qu'interviennent les HSPs chaperonnes.
	De même, il est connu depuis les années 80 que le signal initiateur de la sur-expression de ces protéines est le déséquilibre de l'homéostasie des protéines dénaturées \cite{ananthan1986}.
	Il est donc courrament admis que l'augmentation du nombre de chaperonnes dans un organisme est un indicateur de stress cellulaire \cite{ryan1996}.
	Cet indiquateur est d'autant plus intéressant pour détecter un stress cellulaire que le taux de concervation dans le vivant de certaines HSPs est important.
	En effet, on retrouve des homologies de séquence entre certaines HSPs, à l'échelle de l'ensemble du vivant, ce qui permet leur qualification fréquente de <<~gènes de ménage~>>, et leur utilisation dans des études phylogénétiques concernant l'ensemble du vivant \cite{gupta1995}.

	Les HSPs sont classiquement dénommées et classées dans de grandes familles en fonction de leur poids moléculaire en ångström (par exemple, les protéines de la famille de Hsp70 ont un poids moléculaire d'environ 70\,kDa), mais de récentes tentatives de clarification de cette nomenlature ont abouti à la proposition de nouvelles dénominations pour ces protéines, sans allusion au poids moléculaire \cite{kampinga2009}.

	[Ici, exemple GroEL/ES, si j'ai de la place.]

	\section{Exemples de réponse chez les insectes}

		\subsection{Induction de la réponse de <<~Heat-Shock~>>}

		\paragraph{}
		De nombreux exemples d'études mettant en évidence la réponse <<~Heat-Shock~>> chez l'insecte peuvent être citées.
		Nous en séléctionnerons quelque-unes pour tenter de refléter au mieux la diversité des moyens d'étude et des modèles étudiés.

		% D'abord, le stress chaud.

		% La manip historique : Tessière (J'a pas la publi :( )
		%La première mise en évidence de la réponse au heat-shock par des protéines spécifiques a été réalisée sur des échantillons tissulaires du modèle insecte classique : \esp{Drosophila melanogaster}. Des stress thermiques sont été soumis à des populations

		De nombreuses études ont bien entendu été réalisées sur la drosophile, modèle classique dans les études sur les insectes.
		Outre l'étude historique ayant mis en évidence les HSPs lors d'un choc thermique \cite{tissieres1974}, 
		%des approches quantitatives, moléculaires et enfin protéomiques ont été mises en \oe{}uvre afin de décrire cette réponse heat-shock.
		des approches radicalements différentes ont été mises en \oe{}uvre afin de décrire cette réponse HSP.

		% Feder 1997
		Une approche orientée terrain a été pratiquée par une équipe de chercheurs, dans une étude autour de la réponse de la larve de la drosophile lors du stress thermique induit par une exposition du fruit hôte au soleil \cite{feder1997}.
		Une de leurs expériences a consisté à prélever les larves de drosophiles sur des pommes récoltées à un stade de composition donné, puis laissées à différents degrés d'ensoleillement pendant 2h, à proximité de stations météo afin de contrôles la températude de l'environement.
		Le degré d'expression de la protéine hsp70 utilisée ici comme bioindicateur de la réponse HSP a été mesuré par une méthode ELISA avec un anticorps anti-hsp70.
		Les résultats de cette expérience ont montré une augmentation de 1,5\,\% (32\,\degres{}C) à plus de 48\,\% (38\,°C) de l'expression de hsp70 par rapport à l'expression mesurée dans des conditions physiologiques (25°C).

		% Malmendal 2006 : metabolomique

		% HSPs et stress froid
		%\cite{zhang2011}
		% Ausi Colinet 2010 qui parle des HSP et chill-coma cez droso

		\subsection{Autres types de réponses} % À reformuler.

