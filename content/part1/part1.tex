\chapter{Réponses de l'insecte au stress de température} % (fold)
\label{chap:repstress}
	
	\section{Généralités} % (fold)
	\label{sec:g_n_ralit_s}

		\subsection{Adaptation vs acclimatation} % (fold)
		\label{sub:adaptation_vs_accilmatation}
%			Avant de rentrer plus profondément dans le sujet, il est nécessaire de
rappeler une confusion souvent commise lorsque l'on discute d'une réponse à un
stress. Il s'agit de faire une différence entre la notion d'acclimatation et
celle d'adaptation, souvent confondues dans la langage courant. La différence
se fait essentiellement sur l'échelle de temps durant laquelle se déroule le
phénomène.

Dans le cas de l'adaptation, il s'agit d'une échelle de temps relativement
longue, durant laquelle les organismes seront confrontés à un changement de
condition environementale, induisant une pression de séléction différente des
conditions initiales. De cela va découler une séléction, et une adaptation de
la lignée par modification séléctive des gènes les plus adaptés. L'adaptation
est donc une modification stable et héritable à un stress.

Dans le cas d'une acclimatation, nous sommes dans le cas d'une échelle de
temps beaucoup plus réduite. Les organismes se retrouvent confrontés à une
situation stressante, et va, plus ou moins efficacement, exprimer un phénotype
d'adaptation, que l'on peut assimiler à une resistance à cet environement
stressant. Il s'agit d'une réponse graduelle et souvent réversible, opérée à
l'échelle de l'individu.

\begin{note}
	Commenter l'hypothèse qu'une adaptation serait une suite logique à l'acclimatation, lorsque le stress dure longtemps ?
\end{note}

% Paragraphe sur ce que l'on dit, nous

Dans ce mémoire, nous allons nous intéresser plus en détail aux phénomènes
d'acclimatation suite à un stress ponctuel, exercé à l'échelle des individus.
Cependant, il ne faut pas oublier que l'adaptation génomique à un stress à
long terme peut être une piste également pertinente pour l'étude de la
dynamique des espèces invasives.

Pour reprendre le cas de la résistance au froid, évoqué en introduction, nous
pourrions évoquer la piste des molécules anti-gel. L'utilisation de ces
molécules se retrouvant à la fois dans le groupe des insectes\cite{duman2001}
que dans celui des bactéries\cite{xu1998}, nous pouvons aisément supposer des
intéractions, d'autant plus que certaines de ces molécules sont extrêmement
simples et ubiquistes (petite peptides, petits sucres).


Avant de rentrer plus profondément dans le sujet, il me semble nécessaire de
préciser une subtilité de vocabulaire utile lorsque l'on discute d'une réponse
à un stress. Il s'agit de faire une différence entre la notion d'acclimatation
et celle d'adaptation évolutive, souvent confondues dans la langage courant.
La différence se fait essentiellement sur l'échelle de temps durant laquelle
se déroule le phénomène.

Dans le cas de l'adaptation, il s'agit d'une échelle de temps relativement
longue, durant laquelle les organismes seront confrontés à un changement de
condition environementale, induisant une pression de séléction différente des
conditions initiales. De cela va découler une séléction, et une adaptation de
la lignée par modification séléctive des gènes les plus adaptés. L'adaptation
est donc une réponse stable et héritable à un stress.

Dans le cas d'une acclimatation, nous sommes dans le cas d'une échelle de
temps beaucoup plus réduite. Les organismes se retrouvent confrontés à une
situation stressante, et va, plus ou moins efficacement, exprimer un phénotype
d'adaptation, que l'on peut assimiler à une resistance à cet environement
stressant. Il s'agit d'une réponse graduelle et souvent réversible, opérée à
l'échelle de l'individu. La réponse acclimative est donc le reflet de la
plasticité adaptative d'un individu (ou d'un holobionte, dans le cas qui nous
intéresse).

Dans ce mémoire, nous allons nous intéresser plus en détail aux phénomènes
d'acclimatation suite à un stress ponctuel, exercé à l'échelle des individus.
Cependant, il ne faut pas oublier que l'adaptation génomique à un stress à
long terme peut être une piste également pertinente pour l'étude de la
dynamique des espèces invasives.

Pour reprendre le cas de la résistance au froid, évoqué en introduction, nous
pourrions citer comme réponse adaptative la piste des molécules anti-gel.
L'utilisation de ces molécules se retrouvant à la fois dans le groupe des
insectes\cite{duman2001} que dans celui des bactéries\cite{xu1998}, nous
pouvons aisément supposer des intéractions, d'autant plus que certaines de ces
molécules sont extrêmement simples et ubiquistes (petits peptides, petits
sucres).


%		\subsection{Quelques exemples d'adaptation aux températures extrêmes} % (fold)
%		\label{sub:exemples_adaptations}
%			\begin{note}

C'est pas certain que l'on garde cette partie, un peut HS.

Ici c’est pour parler de protéines anti-gel, de trucs anti-dessication pour les temp chaudes\ldots

\end{note} % Huhu, EndNote ! lol…

		\subsection{Les Heat-Shock Proteins (HSP), clé de voûte de la réponse acclimative} % (fold)
		\label{sub:generalites_HSP}
%			\cite{federhoffmann1999, zhang2011}

Le terme HSP (\eng{Heat Shock Proteins}) désigne une grande famille de
protéines initialement décrites comme étant induite spécifiquement par un
stress thermique chaud. Par la suite, elles se sont avérées jouer un rôle bien
plus large dans la réponse générique aux stress de diverses
natures\cite{sorensen2003}\footnote{Tableau Sørensen pour illustrer}.

% Rôles

Ces protéines font pour la plupart partie du grand groupe des protéines chaperonnes. Les
protéines chaperonnes jouent des rôles multiples tournent tous autour de la
gestion de la conformation des protéines, de leur repliement, leur appariement
dans des complexes protéiques, ou encore le transport.  Leur rôle est crucial
dans le processus de formation de protéines, y compris dans un environement
non stressant. Cependant, lors d'un stress, le besoin de réparation rapide des
structures endomagées par ce stress se fait plus fort, d'où l'existance de
cette sous-famille de chaperonnes que sont les HSPs.

% Fort taux de concervation

Les HSPs ont un taux de concervation dans le vivant extrêmement important, ce
qui fait que leurs gènes sont souvent califiées de gènes dits <<~de ménage~>>
(\eng{“Housekeeping genes”} en anglais)

% Familles

Ces HSPs sont classiquement classées dans des grandes familles en fonction de
leur poids moléculaire (par exemple, les protéines de la famille de Hsp70 ont
un poids moléculaire d'environ 70\,kDa), mais de récentes tentatives de
clarification de cette nomenlature tendraient vers des noms ne faisant pas
allusion au poids moléculaire\footnote{Kampinga et al., 2009. À voir si je le
cite ou pas, juste pour ça...}

\subsubsection{Les petites HSPs} % (fold)
\label{ssub:les_petites_hsps}

  Les petites HSPs, dont le poids moléculaire se situe entre 12 et 43\,kDa,
  contrairement aux autres familles ne sont exprimmées que lors d'un choc
  thermique. Leur rôle est assez mal connu\footnote{Mettre à jour, peut-
  être...}, mais elles sont décrites comme se liant aux protéines dénaturées.
  Le rôle de cette liaison peut être celui d'empêcher les protéines dénaturées
  de s'aggréger, ce qui gênerait l'action des HSPs chaperonnes.

\subsubsection{La famille HSP40} % (fold)
\label{ssub:la_famille_hsp40}

  La famille HSP40, aussi appelée DnaJ, contient des protéines hautement
  concervées, carractérisées par la présence du domaine J, qui correspond au
  domaine hautement concervé de la protéine. Le rôle le plus décrit des
  protéines de ce groupe est celui de cochaperonnes pour HSP70. D'autres rôles
  sont mis en évidence dans la littérature, comme un rôle dans l'adressage à
  l'ubiquitine \cite{lee1996}.

\subsubsection{La famile HSP60} % (fold)
\label{ssub:la_famile_hsp60}

La famille HSP60, aussi appelées chaperonines (cpn60), comprend deux grands
complexes protéiques : GroEL et TCP-1. GroEL et ses homologues se retrouvent
dans les organismes procaryotes (dont la mitochondrie et le chloroplaste),
alors que TCP-1 est exprimmé dans le cytoplasme de la cellule eucaryote.

Dans la bactérie, la présence de la co-chaperonine GroES (couplée à l'ATP) est
nécessaire pour le fonctionnement de GroEL.

Certains membres de cette famille entrent aussi dans la constitution de la
Ribulose-1,5-bisphosphate carboxylase oxygenase (RuBisCO), protéine centrale
du processus de photosynthèse chez les plantes.

\subsubsection{La famille HSP70} % (fold)
\label{ssub:la_famille_hsp70}

Les protéines de la famille HSP70 sont très nombreuses, et ont souvent
plusieurs représentants dans un seul organisme (par exemple, la plupart des
eucaryotes en ont plus de 10 différents, répartis dans tous les compartiments
cellulaires). Elles jouent un rôle de chaperones, associées aux co-chaperonnes
DnaJ et GrpE. Nous en retiendrons une qui semble plus que les autres induite
par le stress, BiP (ou Grp78), une HSP70 cytoplasmique.

\subsubsection{La famille HSP90} % (fold)
\label{ssub:la_famille_hsp90}

\subsubsection{La famille HSP100} % (fold)
\label{ssub:la_famille_hsp100}


\input{content/figures/tab_HSP}


\cite{federhoffmann1999, zhang2011}

Le terme HSP (\eng{Heat Shock Proteins}) désigne une grande famille de
protéines initialement décrites comme étant induite spécifiquement par un
stress thermique chaud. Par la suite, elles se sont avérées jouer un rôle bien
plus large dans la réponse générique aux stress de diverses
natures\cite{sorensen2003}\footnote{Tableau Sørensen pour illustrer (Sørensen 2003, p. 3)}.

% Rôles

Ces protéines font pour la plupart partie du grand groupe des protéines chaperonnes. Les
protéines chaperonnes jouent des rôles multiples tournent tous autour de la
gestion de la conformation des protéines, de leur repliement, leur appariement
dans des complexes protéiques, ou encore le transport.  Leur rôle est crucial
dans le processus de formation de protéines, y compris dans un environement
non stressant. Cependant, lors d'un stress, le besoin de réparation rapide des
structures endomagées par ce stress se fait plus fort, d'où l'existance de
cette sous-famille de chaperonnes que sont les HSPs.

% Fort taux de concervation

Les HSPs ont un taux de concervation dans le vivant extrêmement important, ce
qui fait que leurs gènes sont souvent califiées de gènes dits <<~de ménage~>>
(\eng{“Housekeeping genes”} en anglais).

% Familles

Ces HSPs sont classiquement classées dans des grandes familles en fonction de
leur poids moléculaire \cite{fink1999} (par exemple, les protéines de la
famille de Hsp70 ont un poids moléculaire d'environ 70\,kDa), mais de récentes
tentatives de clarification de cette nomenlature tendraient vers des noms ne
faisant pas allusion au poids moléculaire\footnote{Kampinga et al., 2009. À
voir si je le cite ou pas, juste pour ça...}

% Les HSPs sont régulées par heat shock transcription factor (HSF1) (reviewed in Morimoto 1998).

\subsubsection{Les petites HSPs} % (fold)
\label{ssub:les_petites_hsps}

  Les petites HSPs, dont le poids moléculaire se situe entre 12 et 43\,kDa,
  contrairement aux autres familles ne sont exprimmées que lors d'un choc
  thermique. Leur rôle est assez mal connu\footnote{Mettre à jour, peut-
  être...}, mais elles sont décrites comme se liant aux protéines dénaturées.
  Le rôle de cette liaison peut être celui d'empêcher les protéines dénaturées
  de s'aggréger, ce qui gênerait l'action des HSPs chaperonnes.

  Exemple : \cite{liu2013} (Sur-expression d’une petite HSP, HSP21, protège le
  ver à soie des stress thermiques)

  Il a aussi été montré une forte affinité entre ces petites HSPs et le cytosquelette.
  Les petites HSPs joueraient un rôle dans le maintien du cytoquelette lors d'un stress thermique.


\subsubsection{La famille HSP40} % (fold)
\label{ssub:la_famille_hsp40}

  La famille HSP40, aussi appelée DnaJ, contient des protéines hautement
  concervées, carractérisées par la présence du domaine J, qui correspond au
  domaine hautement concervé de la protéine. Le rôle le plus décrit des
  protéines de ce groupe est celui de cochaperonnes pour HSP70. D'autres rôles
  sont mis en évidence dans la littérature, comme un rôle dans l'adressage à
  l'ubiquitine \cite{lee1996}.

\subsubsection{La famile HSP60} % (fold)
\label{ssub:la_famile_hsp60}

  La famille HSP60, aussi appelées chaperonines (cpn60), comprend deux grands
  complexes protéiques : GroEL et TCP-1. GroEL et ses homologues se retrouvent
  dans les organismes procaryotes (dont la mitochondrie et le chloroplaste),
  alors que TCP-1 est exprimmé dans le cytoplasme de la cellule eucaryote.

  Dans la bactérie, la présence de la co-chaperonine GroES (couplée à l'ATP)
  est nécessaire pour le fonctionnement de GroEL.

  Certains membres de cette famille entrent aussi dans la constitution de la
  Ribulose-1,5-bisphosphate carboxylase oxygenase (RuBisCO), protéine centrale
  du processus de photosynthèse chez les plantes.

\subsubsection{La famille HSP70} % (fold)
\label{ssub:la_famille_hsp70}

  Les protéines de la famille HSP70 sont très nombreuses, et ont souvent
  plusieurs représentants dans un seul organisme (par exemple, la plupart des
  eucaryotes en ont plus de 10 différents, répartis dans tous les
  compartiments cellulaires). Elles jouent un rôle de chaperones, associées
  aux co-chaperonnes DnaJ et GrpE. Nous en retiendrons une qui semble plus que
  les autres induite par le stress, BiP (ou Grp78), une HSP70 cytoplasmique.

\subsubsection{La famille HSP90} % (fold)
\label{ssub:la_famille_hsp90}

\todo[inline]{Citation : \citet{chiosis2013} ssi j’arrive à obtenir un accès (Nature Structural \& Molecular Biology).\\
Les publis que j’ai trouvé jusqu’à maintenant semblent dire que les rôles in vivo de HSP90 sont mal connus, mais elles (les publis) sont un peut vieilles. Dans celle-là, peut-être que je trouverais plus de choses.}

  Les HSP90 sont, à l'instar de HSP40, des protéines hautement concervées. On
  en retrouve dans tous les organismes, qu’ils soient eucaryotes ou
  procaryotes. Elles sont souvent associées à HP70, dans leur rôle de
  chaperonnes. On leur attribue aussi des rôles dans les mécanismes de
  transduction du signal, et en association avec le cytosquelette.

\subsubsection{La famille HSP100} % (fold)
\label{ssub:la_famille_hsp100}

Les HSP100 jouent un rôle important dans la réponse au stress thermique. % 78 in Fink
Elles se conforment, à la manière de HSP60, en forme d'anneau, avec cependant un diamètre inférieur, ce qui laisse supposer un rôle différent de celui de chaperonne de GroEL.
En effet, il a été montré que les HSP100 jouent un rôle dans la protéolyse des protéines dénaturées.

%\input{content/figures/tab_HSP}


	\section{Autres processus impliqués dans la réponse aux stress thermiques, et régulations} % (fold)
	\label{sec:rep_misc}
%		Outre les HSP, nous pouvons citer d'autres phénomènes constituant une réponse au stress thermique chez l'insecte.

Tout d'abord, il est important de rappeler que le stress thermique, avant d'exercer ses effets délétères sur la physiologie, notemment les peotéines, induitdans un premier temps une réaction d'ordre nerveuse et hormonale, qui servira de signal afin de signaler la perturbation.
La dopamine est un eurotrnsmetteur de l'insecte associé entre autres aux signaux de type stress, et est sollicité lors d'un stress thermique \cite{andersen2006}

% Il y a aussi (Armstrong et al), avec cette histoire de K+ dans le cerveau…

\paragraph{}

Les chocs de température n'affectent pas uniquement la conformation des protéines.
Un autre point clé physiologique est aussi sévèrement touché : le système nerveux.
La perturbation termique va affecter les mouvements ioniques (K⁺) nécessaires aux potentiels d'action, va notament provoquer un phénomène de «coma» thermique (\eng{chill coma} en anglais).


Outre les HSP, nous pouvons citer d'autres phénomènes constituant une réponse au stress thermique chez l'insecte.

Tout d'abord, il est important de rappeler que le stress thermique, avant d'exercer ses effets délétères sur la physiologie, notemment les peotéines, induitdans un premier temps une réaction d'ordre nerveuse et hormonale, qui servira de signal afin de signaler la perturbation.
La dopamine est un eurotrnsmetteur de l'insecte associé entre autres aux signaux de type stress, et est sollicité lors d'un stress thermique \cite{andersen2006}

% Il y a aussi (Armstrong et al), avec cette histoire de K+ dans le cerveau…

\paragraph{}

Les chocs de température n'affectent pas uniquement la conformation des protéines.
Un autre point clé physiologique est aussi sévèrement touché : le système nerveux.
Le choc termique va affecter les mouvements ioniques (K⁺) nécessaires aux potentiels d'action, en opérant une perturbation des caneaux ioniques.
Cela va notament provoquer un phénomène de «coma» thermique (\eng{chill coma} en anglais).


	\section{Exemples chez les modèles d'insectes les plus étudiés} % (fold)
	\label{sec:exemples_modeles}
		Nous allons maintenant détailler quelques exemples de cas dans lesquels le
partenaire microbien a un rôle prépondérent dans les mécanismes de réaction au
stress thermique de leur hôte. 


		\subsection{Exemple droso} % (fold)
		\label{sub:exemple_droso}
			\input{content/part1/exemple_droso}

		\subsection{Exemple moustique} % (fold)
		\label{sub:exemple_moustique}
			\input{content/part1/exemple_moustique}

		\subsection{Exemple puceron} % (fold)
		\label{sub:exemple_puceron}
			\input{content/part1/exemple_puceron}

% chapter stressinsect (end)