Lorsque l’on veut étudier la dynamique des espèces invasives, un facteur clé à
prendre en compte est leur capacité à s’adapter à des environements hostiles,
typiquement à des climats différent fortement de ceux d’où provienent ces
espèces. En l’occurence, lorsqu’on s’intéresse aux insectes vecteurs de
maladies virales, comme le moustique tigre \esp{Aedes albopictus} (vecteur de
la Dengue, espèce invasive d’origine tropicale), la capacité à s’adapter aux
climats tempérés, voire froids, est cruciale.

Nous allons nous intéresser à cette problématique, tout d’abors en résumant
l’état des connaissances sur les mécanismes impliqués dans la résistance au
froid chez les insectes, puis en nous intéressant plus spécifiquement au rôle
que peuvent avoir les micro-organismes symbiotiques dans cette capacité
d’adaptation.
