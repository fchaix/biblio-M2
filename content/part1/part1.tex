\chapter{Réponses de l'insecte aux stress de température}

% Préambule acclimatation vs. adaptation
En biologie tout stress appelle une réponse de l'organisme pour maintenir l'équilibre biologique dans un état fonctionnel.
Deux types de réponses, souvent confondues à tort, sont observées lors d'un stress : l'adaptation et l'acclimatation.
La différence se situe essentiellement sur l'échelle de temps durant laquelle se déroule le phénomène.
Dans le cas de l'adaptation, les organismes seront confrontés à un changement environnemental sur l'échelle de plusieurs générations.
Ce changement sur le long terme va introduire une pression de sélection qui va à plus ou moins long terme modifier le génome de la lignée (ou la composition de l'holobionte), qui sera plus adaptée à ce nouvel environnement.
Dans le cas de l'acclimatation, l'échelle de temps est réduite à celle de la vie d'un individu. La réponse au changement environnemental sera dépendant des capacités intrinsèques de l'individu ou de l'holobionte en question, et ces changements seront réversibles.
On parle aussi de plasticité adaptative de l'individu pour désigner cette capacité d'acclimatation.
Ce mémoire sera plus particulièrement consacré à l'étude des
phénomènes d'acclimatation suite à un stress ponctuel dans le temps, exercé à l'échelle des individus.

	\section{Les protéines de choc thermique, clé de voûte de la réponse acclimative}

	La famille de protéines de réponse au stress que nous allons évoquer dans cette partie ne concerne pas uniquement les insectes.
	Les HSP (\eng{Heat Shock Proteins}) constituent en effet la réponse la plus universelle au stress thermique qu'il est possible de trouver, des bactéries aux vertébrés supérieurs.
	Elles sont par conséquent de loin les plus documentées dans la littérature.
	Leur découverte remonte à 1974 où Tissière \textit{et al.} les mit en évidence par électrophorèse dans des extraits tissulaires de drosophiles ayant subi un stress chaud \cite{tissieres1974}.
	Par la suite, elles se sont avérées jouer un rôle bien plus large dans la réponse générique aux stress de diverses natures, comme les radiations aux UV, l'hypoxie et d'autres stress, listés dans le tableau 1 \cite{sorensen2003}.

	Ces protéines appartiennent pour la plupart au grand groupe des protéines chaperonnes \cite{federhoffmann1999}.
	Celles-ce jouent des rôles multiples, en particulier celui de la gestion de la conformation des protéines, leur appariement dans des complexes protéiques, ou encore dans le transport.
	Leur rôle est crucial dans le processus de formation de protéines, y compris dans des conditions environnementales non stressantes.
	Mais quand survient un stress le besoin accru de réparation rapide des structures endommagées est nécessaire, c'est donc en réponse à ce besoin qu'interviennent les HSPs chaperonnes.
	De même, il est connu depuis les années 80 que le signal initiateur de la sur-expression de ces protéines est le déséquilibre de l'homéostasie des protéines dénaturées \cite{ananthan1986}.
	Il est donc couramment admis que l'augmentation du nombre de chaperonnes dans un organisme est un indicateur de stress cellulaire \cite{ryan1996}.
	Cet indicateur est d'autant plus intéressant pour détecter un stress cellulaire que le taux de conservation dans le vivant de certaines HSPs est important.
	En effet, on retrouve des homologies de séquence entre certaines HSPs, à l'échelle de l'ensemble du vivant, ce qui permet leur qualification fréquente de <<~gènes de ménage~>>, et leur utilisation dans des études phylogénétiques concernant l'ensemble du vivant \cite{gupta1995}.

	Les HSPs sont classiquement dénommées et classées dans de grandes familles en fonction de leur poids moléculaire en ångström (par exemple, les protéines de la famille de Hsp70 ont un poids moléculaire d'environ 70\,kDa), mais de récentes tentatives de clarification de cette nomenclature ont abouti à la proposition de nouvelles dénominations pour ces protéines, sans allusion au poids moléculaire \cite{kampinga2009}.

	[Ici, exemple GroEL/ES, si j'ai de la place.]

	\section{Exemples de réponse chez les insectes}

		\subsection{Induction de la réponse de <<~Heat-Shock~>>}
		% ↑ Peut-être titre à changer, on parle aussi pas mal des moyens d'étude...
		\paragraph{}
		De nombreux exemples d'études mettant en évidence la réponse <<~Heat-Shock~>> chez l'insecte peuvent être citées.
		Nous en sélectionnerons quelque-unes pour tenter de refléter au mieux la diversité des moyens d'étude et des modèles étudiés.

		% D'abord, le stress chaud.

		% La manip historique : Tessière (J'a pas la publi :( )
		%La première mise en évidence de la réponse au heat-shock par des protéines spécifiques a été réalisée sur des échantillons tissulaires du modèle insecte classique : \esp{Drosophila melanogaster}. Des stress thermiques sont été soumis à des populations

		De nombreuses études ont bien entendu été réalisées sur la drosophile, modèle classique dans les études sur les insectes.
		Nous essaierons cependant de citer des études sur des modèles variés, ainsi que des types de chocs thermiques différents, afin d'illustrer l'universalité de ce type de réponse.
		Outre l'étude historique ayant mis en évidence les HSPs lors d'un choc thermique \cite{tissieres1974}, 
		%des approches quantitatives, moléculaires et enfin protéomiques ont été mises en \oe{}uvre afin de décrire cette réponse heat-shock.
		des approches radicalement différentes ont été mises en \oe{}uvre afin de décrire cette réponse HSP.

		% Feder 1997
		Une approche orientée terrain a été pratiquée par une équipe de chercheurs, dans une étude autour de la réponse de la larve de la drosophile lors du stress thermique induit par une exposition du fruit hôte au soleil \cite{feder1997}.
		Une de leurs expériences a consisté à prélever les larves de drosophiles sur des pommes récoltées à un stade de composition donné, puis laissées à différents degrés d'ensoleillement pendant 2h, à proximité de stations météo afin de contrôles la température de l'environnement.
		Le degré d'expression de la protéine hsp70 utilisée ici comme bioindicateur de la réponse HSP a été mesuré par une méthode ELISA avec un anticorps anti-hsp70.
		Les résultats de cette expérience ont montré une augmentation de 1,5\,\% (32\,\degres{}C) à plus de 48\,\% (38\,°C) de l'expression de hsp70 par rapport à l'expression mesurée dans des conditions physiologiques (25°C). % Y'a un souci avec le signe degré et la police helvetica.

		Parmi les méthodes plus modernes permettant de mettre en évidence une réponse cellulaire d'un choc thermique figure les techniques de protéomique.
		L'exemple de ce type d'études que nous allons citer concerne la guêpe \esp{Aphidius colemani}, un parasitoïde du puceron souvent utilisé dans la lutte biologique contre ce ravageur \cite{colinet2007}.
		L'étude compare la réponse protéomique de la guêpe confrontée soit à un environnement constant et froid, soit à un environnement alternant des périodes froides avec de courts événements chauds, dans le but de comprendre la réponse protéomique de l'insecte à un stress froid.
		Le stade de développement soumis au stress est le stade nymphal.
		Pour chaque condition, une extraction des protéines puis une séparation par éléctrophorèse bidimentionelle ont été effectuées.
		Les spots différents du contrôle (resté à température ambiante) sont carractérisés par spectrométrie de masse, puis comparée à des bases de données.
		De nombreuses protéines ont été ainsi mises en évidence comme étant surexprimées dans la condition à températude variable, parmi lesquelles figurent des HSPs (hsp70, hsp90) ainsi que des protéines impliquées dans le métabolisme énergétique.

		% Cinétique réponse HS → Nguyen et al 2009

		% Zhao2009 : Identification des gènes exprimés diférement lors d'un HS
		Enfin, une autre approche que l'on pourrait citer afin de mettre en évidence des différences d'expression de gènes lors d'un stress thermique consisterait en une analyse de l'ARN transcrit en des quantités différentes du contrôle, par la technique de <<~Suppression subtractive hybridization~>>\footnote{CVM : Je ne sais pas coment traduire le nom de cett technique, si il existe une traduction}.
		Cette technique est une méthode de PCR qui amplifie spécifiquement les ADN complémentaires (obtenus à partir de l'ARN extrait des insectes soumis ou non au stress thermique) qui diffèrent de ceux di contrôle.
		Cela permet donc de détecter les régulations transcriptionnelles induites par le stress.
		L'étude menée ainsi sur le moustique \esp{Aedes aegypti}, un vecteur notamment du virus de la Dengue et du Chikungunya, révelle ainsi une augmentation de l'expression de certaines HSPs, et d'autres protéines (synthèse protéique, métabolisme énergétique) suite à un stress thermique chaud.

		% HSPs et stress froid
		%\cite{zhang2011}
		% Ausi Colinet 2010 qui parle des HSP et chill-coma cez droso

		\subsection{Autres types de réponses physiologiques au stress thermique} % À reformuler.

		\paragraph{}
		Les réponses physiologiques à un stress abiotique aussi important que la température ne peuvent pas être réduite à la seule réponse de la stimulation de la production de HSPs.
		Nous donnerons dans cette partie quelques exemples, en sachant que l'on ne pourra pas être exaustifs.

		\paragraph{}

		% Andersen 2006 (Dopamine levels in the mosquito Aedes aegypti during adult development, following blood feeding and in response to heat stress) → Résultats pas significatifs
		% Hirashima200
		Les amines biogéniques, comme la dopamine ou l'octopamine sont connus comme étant des neurotransmetteur et/ou hormones circulant dans l'hémolymphe, impliqués dans de multiples mécanismes physiologiques de l'insecte, comme la synthèse d'hormones, le processus de mue, ou la locomotion.
		Elle est de plus connue pour être aussi une réponse non-spécifique aux stress, de part ses actions stimulantes.
		% Par exemple, dans le cas de la locomotion, sa stimulation permettrait des changements éthologiques comme une fuite de la source du stress, 
		L'étude que nous donnerons en exemple prend comme modèle la drosophile \esp{Drosophila virilis} \cite{hirashima2000}.
		On fera subir à ces mouches un stress thermique (exposition à 38\,°C, pendant des durées variables), et le taux des différentes amines biogéniques sera mesuré par une technique d'HPLC (\eng{High-Pressure Liquid Chromatography}), a partir d'extraits de tissus de la tête de la drosophile.
		Les résultats montrent une augmentation de ces amines biogéniques, signes d'une réponse nerveuse et humorale de l'insecte au stress de température.

		% Armstrong 2012 : K⁺
		La deuxième réponse que l'on citera, concerne cette fois l'homéostasie du potassium dans l'hémolymphe \cite{armstrong2012}.
		Nous savons que la chûte de la concentration de K\up{+} provoque chez l'insecte un <<~coma~>> temporaire, et que cette chûte de K\up{+} peut être causée par une chûte de température.
		L'étude nous montre qu'un pré-traiement avec des chocs froids successifs <<~immunise~>> la mouche contre ce phénomène.
		Les mécanismes impliqués dans cette acclimatation rapide aux chocs froids sont encore inconnus (bien que l'effet du froid sur les pompes à potassium soient fortement suspectées), mais constituent bien une réponse de l'insecte à un stress froid ponctuel.

		% Métabolomique : Malmendal2006
		Enfin, des études basées sur des téchniques de métabolomiques montrent aussi des réponses de l'insecte au stress thermique, du point de vue du métabolisme \cite{malmendal2006}.

% THV (Pucerons...)
