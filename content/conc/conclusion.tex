
\section*{Conclusion}
\addcontentsline{toc}{section}{Conclusion}

Les réponses aux stress thermiques constituent une problématique complexe, de part la multiplicité des types de réponses. 
Nous n'avons pas, par exemple, évoqué les réponses éthologiques (fuite, migrations, construction d'abris...), et sans doute passé sous silence certains éléments physiologiques.
Cela reflète la grande diversité des réponses à ces stress, et le fait que de part son importance dans les écosystèmes vivants, ces différentes réponses ont déjà été étudiées sous beaucoup d'angles.
Il n'a cependant pas été beaucoup étudié sous l'angle de l'holobionte, notion qui permet pourtant d'ajouter un grand facteur de plasticité dans notre compréhesion de l'interface individu-environement.
Cependant la prise de conscience de cette dimention est aujourd'hui en plein essor, notamment en raison de l'importance grandissante des épidémies d'arboviroses.

C'est justement dans cet esprit que mon stage de master se déroulera, il consistera à mesurer les interférences entre la réponse du moustique tigre \esp{Aedes albopictus} au stress froid et le symbiote secondaire cultivable Acinetobacter calcoaecitus.

