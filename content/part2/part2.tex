\chapter{Interférences du microbiote dans la réponse aux stress de température} % (fold)

\paragraph{}
Nous avons vu différentes voies de réponse de l'insecte aux stress thermiques, en réduisant le système à l'insecte seul.
Or nous savons que dans la nature, la séléction s'exerce sur l'ensemble formé par l'hôte et son microbiote, il s'agit de la notion d'holobionte, défini précédement.
Dans un système en constante intéraction entre les partenaires, il est intuitif de supposer que le microbiote puisse jouer un rôle, plus ou moins directement, dans la réponse au stress thermique, de laquelle dépend la fitness de l'holobionte ?

La diversité des mécanismes pouvant mettre en relation le microbiote bactérien et leur hôte dans un phénomène de réaction à un stress thermique est très grande, et fait surtout l'objet de recherches toujours actives.
Nous décrirons dans cette partie les mécanismes les plus fréquemment retrouvées à l'heure actuelle.

\section{Stimulation de l'expression de gènes de l'hôte}

Dans ce type de relation, la présence de la bactérie dans les tissus de l'insecte stimule l'expression de protéines de stress, ou du moins de thermotolérance, quelque soit le contexte environemental.
En effet, l'infection bactérienne est connue comme étant un déclancheur d'une réponse de heat shock (voir tableau 1).
L'hôte est donc dans un état de stress <<~artificiel~>>, y compris dans un environnement non stressant, ce qui lui permet d'être <<~préparé~>> à un éventuel choc thermique, et donc de mieux y résister.

Notre premier exemple concernera le couple symbiotique que forme la mouche \esp{Bemisia tabaci}, un ravageur de culture et vecteur de virus phytopathogènes, et son symbiote secondaire \esp{Rickettsia sp.} \cite{brumin2011}.
Cette étude a mis en évidence un mécanisme conduisant à une meilleure résistance de la mouche au stress thermique, lorsqu'elle est associée à \esp{Rickettsia sp.}.
Partant du constat que les populations associées à \esp{Rickettsia sp.} étaient plus résistants aux chocs thermiques chauds (incubation des femelles adultes à différentes températures, pendant 3h30), les chercheurs ont mesuré l'expression de gènes connus comme faisant partie de la réponse thermotolérante.
Les résultats montrent que les HSPs, contrairement à ce à quoi l'on pourrait s'attendre, ne sont pas significativement induits par la présence de \esp{Rickettsia sp.}, mais qu'en revanche diverses protéines associées au cytosquelette sont nettement induites.
Or, on sait que le cytosquelette, de part sa structure protéique, est l'un des organes cellulaire le plus rapidement touché par les chocs thermiques.
\esp{Rickettsia sp.} semblerait capable d'induire la sur-expression de protéines du cytosquelette de leur hôte, induisant ainsi une meilleure résistance aux stress thermiques.

%%%% Les publis que j'ai pas encore mis
\cite{tang2012}
\cite{dunbar2007}
\cite{stoll2009}
\cite{harmon2009}
\cite{montllor2002}
\cite{bordenstein2011}
