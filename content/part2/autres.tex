D'autres mécanismes mettant en jeu lamicroflore bactérienne peuvent être
observés, influant plus ou moins directement la résistance aux stress
thermiques.

\subsection{Variation de la composition et de la taille de la population symbiotique} % (fold)
\label{sub:taillepop}

Les contitions climatiques sont souvent corrélées à un changement parfois
raducal dans la nature de la population bactérienne associée à l'hôte. Il a
par exemple été montré \cite{harmon2009} que les populations de pucerons
issues de milieux soumis à des chocs thermiques réguliers présentaient une
prévalence de leurs symbiontes secondaires bien plus élevés que ceux de pays
plus cléments.

\subsection{Relations entre les différentes populations symbiotiques au sein d'un même hôte} 
\label{sub:sociologie_des_symbiotes}

Il n'est pas rare qu'au sein d'un même hôte plusieurs populations de symbiotes
cohabitent. Il se met alors en place des intéractions classiques entre ces
populations, pouvant aller de l'antagonisme au commensalisme. Par exemple, il
a été montré \cite{montllor2002} que la présence de symbiotes secondaires
comme \esp{Rickettsia sp.} protégeait les bactériocytes\footnote{Les
bactériocytes sont des cellules de l'hôte spécialisées dans la relation avec
son endosymbiote, dans le cas du puceron, \esp{Bushnera sp.}} du puceron,
lorsque celui-ci est soumis à un choc thermique.

\subsection{diminution antagonismes} % (fold)
\label{sub:diminution_antagonismes}

→ \esp{Wolbachia} et diminution de l'IC pendant ub HS.
\todo[inline]{Essayer de trouver le fulltext de \citet{arakaki2001}, qui parle de ça. Sinon, on vire la partie (ou on trouve une autre publi là-dessus)}