\subsection{Processus mis en jeu} % (fold)
\label{sub:processus_mis_en_jeu}

Dans ce type de relation, la présence de la bactérie dans les tissus de
l'insecte stimule l'expression de protéines de stress, quelque soit le
contexte environemental. Les protéines de stress sont donc exprimées de façon
<<artificiellemnt>> constitutive, ce qui permet à l'insecte de résister de façon
plus efficace à un stress brutal et rapide.

\subsection{Exemples} % (fold)
\label{sub:exemples}

% Exemple Rickettsia (brumin2011)

Notre premier exemple concernera le couple symbiotique que forme la mouche
\esp{Bemisia tabaci}, un ravageur de culture et vecteur de virus
phytopathogènes, et son symbiote secondaire \esp{Rickettsia sp.}. Une étude
\cite{brumin2011} a récement mis en évidence un mécanisme conduisant à une
meilleure résistance de la mouche au stress thermique, lorsqu'elle est
associée à sa bactérie symbiotique. Le processus mis en jeu dans cette
résistance est celui décrit en \ref{sub:processus_mis_en_jeu}, a savoir que la
présence de la bactérie dans les tissus de la mouche induit une expression de
gènes nécessaires à la thermotolérance (notamment ici des gènes codant pour
des éléments du cytosquelette), même dans des insectes non-soumis à un stress
thermique.

% Exemple Bushnera et mutation gène ibpA (dunbar2007)