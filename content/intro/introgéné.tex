%%%%%%%% 1er paragraph %%%%%%%%%%%%%%%%%%%%%%%%
% 1ère partie : expose l'aspect général du sujet (CONNU)
%%%%%%%%%%%%%%%%%%%%%%%%%%%%%%%%%%%%%%%%

% Notes : À placer : poikilotherms (inverse de l'homéotherme)

\paragraph{} % (fold)
\label{par:intro1}

Tous les organismes dans la nature sont sans cesse confrontés à des stress abiotiques, et doivent résister à ceux-ci.
Le stress thermique en particulier a été tres étudié, car représentant un stress à la fois concernant tous les organismes et facile à mettre en oeuvre de manière expérimentale, ce qui en a fait un type de stress que l'on pourrait califier de <<~modèle~>> dans le cadre des études du stress en général.
Les insectes en particulier, par leur qualité de poïkilothermes\footnote{Un organisme poïkilotherme, par opposition avec les homéothermes, est un organisme qui voit naturellement sa température interne varier très fortement en fonction de celle de son environnement.},
sont particuliérement sensibles aux changements brutaux de température.
On retrouvera donc au sein de ce groupe des capacités d'adaptation aux stress thermiques qui, de surcroît, ont été beaucoup étudiées, en raison des fortes implications des insectes dans des problématiques d'importance économiques (ravageurs, espèces invasives…) ou touchant la santé publique (insectes vecteurs…).

%
	% Il est de plus en plus admis dans la communauté scientifique que le microbiote
	% associé à un hôte, que l'on parle de flore commensale ou symbiotique
	% \footnote{Le terme de symbiote sera utilisé ici au sens lagre, a savoir
	% comprennant à la fois les symbiotes au sens strict (bénéfice réciproque), mais
	% aussi les parasites, le point dommun étant le carractère obligatoire de la
	% relation qu'il entretient avec son hôte.}, a un impact important sur les
	% traits d'histoire de vie de son hôte \cite{feldhaar2011}. De ce constat
	% découle la notion d'holobionte \cite{rosenberg2007}, qui consiste à considérer
	% en tant qu'unité séléctve non plus le génotype d'une espèce, mais ceux,
	% combinés, de toute sa microflore et de lui-même, formant ainsi une sorte de
	% méta-génome, et potentiellement de méta-organisme sur lequel la pression de
	% sélection s'éxercerait.

	% Pour prendre l'exemple des insectes, nous pouvons par exemple citer le cas de
	% la bactérie \esp{Buchnera sp.},  endosymbiotique du puceron, jouant un rôle
	% essentiel dans le métabolisme de son hôte,  en synthétisant des acides aminés
	% que le puceron n'arriverait pas à obtenir en quantité suffisante dans la sève
	% élaborée, qui constitue sa seule source de nourriture \cite{douglas1998}.
	% \todo[inline]{Rédiger exemple du puceron qui change de couleur avec un changement de symbiote (impact sur la rédation), c’est un exemple rigolo}


%%%%%%%% 2ème paragraph %%%%%%%%%%%%%%%%%%%%%%%
% 2ème partie : précise l'aspect particulier du problème (VA VERS L'INCONNU)
%%%%%%%%%%%%%%%%%%%%%%%%%%%%%%%%%%%%%%%%

\paragraph{} % (fold)
\label{par:intro2}

Parmi les approches permettant d'essayer de comprendre les mécanismes de réponse aux stress, il en est une de plus en plus en vogue : celle consistant à prendre en compte la communauté microbienne associée à l'insecte, comme partie prennante de la réponse à l'environement.
En effet, Il est de plus en plus admis dans la communauté scientifique que le microbiote associé à un hôte, que l'on parle de flore commensale ou symbiotique%
\footnote{Le terme de symbiote sera utilisé ici au sens lagre, a savoir comprennant à la fois les symbiotes au sens strict (bénéfice réciproque), mais aussi les parasites, le point dommun étant le carractère obligatoire de la relation qu'il entretient avec son hôte.},
a un impact important sur les traits d'histoire de vie de son hôte \cite{feldhaar2011}.
De ce constat découle la notion d'holobionte \cite{rosenberg2007}, qui consiste à considérer en tant qu'unité séléctve non plus le génotype d'une espèce, mais ceux, combinés, de toute sa microflore et de lui-même, formant ainsi une sorte de méta-génome, et potentiellement de méta-organisme sur lequel la pression de sélection s'éxercerait.

Pour prendre l'exemple des insectes, nous pouvons par exemple citer le cas de la bactérie \esp{Buchnera sp.},  endosymbiotique du puceron, jouant un rôle essentiel dans le métabolisme de son hôte,  en synthétisant des acides aminés que le puceron n'arriverait pas à obtenir en quantité suffisante dans la sève élaborée, qui constitue sa seule source de nourriture \cite{douglas1998}.


%
	% Parmi les insectes, justement, se retrouve bon nombre d'espèces considérées
	% comme invasives, voir extrêmement invasives comme le moustique tigre
	% \esp{Aedes albopictus}. Leur carractère invasif est extrêmement lié à une
	% capacité à s'acclimater à une variation climatique brutale, car la température
	% environementale est l'un des facteurs abiotiques environementaux affectant le
	% plus les insectes.

	% % En plus d'être invasifs, ils sont méchants. 

	% L'intérêt de l'étude du déterminisme de cette capacité d'adaptation est
	% d'autant plus important que ces insectes invasifs sont parfois vecteurs
	% maladies, comme c'est le cas d'\esp{Aedes albopictus}, cité précédement, ce
	% qui fait de l'étude de ces insectes une problématique majeure de santé
	% publique \cite{schaffner2013}.

%%%%%%%% 3ème paragraph %%%%%%%%%%%%%%%%%%%%%%%%
% 3ème partie : indique le but du travail et un aperçu des résultats 
% (POSE LA QUESTION)
%%%%%%%%%%%%%%%%%%%%%%%%%%%%%%%%%%%%%%%%%%

\paragraph{} % (fold)
\label{par:intro3}

Il serait donc intéressant d'envisager l'étude de cette résistance au stress thermique sous un nouvel angle, celui qui considère comme unité évolutive non plus l'insecte seul, mais l'ensemble que forme l'hôte insecte et ses bactéries symbiotiques et commensales.
Autrement dit, nous chercherons à déterminer l'impact qu'a la µflore des insectes sur leurs capacités de résistance au stress thermique.

%
	% Il serait donc intéressant d'envisager l'étude de cette résistance au stress
	% thermique sous un nouvel angle, celui qui considère comme unité évolutive non
	% plus l'insecte seul, mais l'ensemble que forme l'hôte insecte et ses bactéries
	% symbiotiques et commensales. Autrement dit, nous chercherons à déterminer
	% l'impact qu'a la µflore des insectes sur leurs capacités de résistance au
	% stress thermique.

	% Nous décrirons dans un premier temps les mécanismes de réponse au stress
	% thermique propres à l'insecte, puis dans un second temps, 
	% nous décrirons les principaux mécanismes par lesquels le microbiote peut influer sur cette réponse, avec des exemples pour chaque mécanisme.

	% \todo[inline]{Intro à étoffer}

