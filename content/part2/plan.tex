\chapter{Implication du microbiote dans la réponse au stress de température} % (fold)
\label{sec:implicationµbiote}
	
	\section{Exemples clés} % (fold)
	\label{sec:exemples}
		Nous allons maintenant détailler quelques exemples de cas dans lesquels le
partenaire microbien a un rôle prépondérent dans les mécanismes de réaction au
stress thermique de leur hôte. 


		\subsection{\esp{Rickettsia sp.} et les aleurodes du tabac (\textit{whiteflies})} % (fold)
		\label{sub:rickettsia_et_les_aleurodes}
			\paragraph{description} % (fold)
\label{par:description_whitefly}
\cite{brumin2011}

\paragraph{Processus mis en jeu} % (fold)
\label{par:process_whitefly}


			% Via feldhaar2011

		\subsection{\esp{Buchnera sp.} et le puceron \esp{Acyrthosiphon pisum}} % (fold)
		\label{sub:exemple_buchnera}
			\paragraph{Description} % (fold)
\label{par:buchnera_description}


\cite{dunbar2007}

\paragraph{Processus mis en jeu} % (fold)
\label{par:buchnera_process}

La bactérie sur-exprime la protéine chaperonne groEL (protéine de stress
thermique bactérienne), qui diffuse dans l'hémolymphe de l'hôte. Une fois dans l'hémolymphe, groEL peut exercer sa fonction chaperonne sur les protéines insectes, ce qui confère à l'hôte une aire pour résister au choc thermique.


		\subsection{Rôle de groEL dans l'hôte \esp{Nezara viridula}} % (fold)
		\label{sub:groel}
			\paragraph{Description} % (fold)
\label{par:description_groel}

% \cite{wilcox2003}
 \cite{stoll2009}

\paragraph{mécanismes mis en jeu} % (fold)
\label{par:process_groel}


	\section{Diversité des mécanismes} % (fold)
	\label{sec:diversit_des_m_canismes}
		\input{content/part2/diversité_mécanismes}

		\subsection{Stimulation de l'expression de gènes de l'hôte} % (fold)
		\label{sub:gènes_hôte}
			Exemple : partie \ref{sub:rickettsia_et_les_aleurodes} et \ref{sub:exemple_buchnera}

		\subsection{Utilisation par l'hôte de protéines microbiennes} % (fold)
		\label{sub:gènes_microbiens}
			\input{content/part2/gènes_bactos}

		\subsection{Effet des variations de la taille des populations microbiennes sous l'effet de la température} % (fold)
		\label{sub:pop_bact}
			\input{content/part2/variations_pop_bact}

		\subsection{Diminution des effets antagonistes d'une bactérie parasite} % (fold)
		\label{sub:diminution_antagonisme}
			% Exemple : Wolbachia et IC /parthenogenèse (via Brumin 2011)
% Expliquable par la faible survie dans l'hôte à cause du HS. (Buchner 1965, via Brumin 2011)

