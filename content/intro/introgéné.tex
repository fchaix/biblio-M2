%%%%%%%% 1er paragraph %%%%%%%%%%%%%%%%%%%%%%%%
% 1ère partie : expose l'aspect général du sujet (CONNU)
%%%%%%%%%%%%%%%%%%%%%%%%%%%%%%%%%%%%%%%%

\paragraph{} % (fold)
\label{par:intro1}

La communauté scientifique se rend de plus en plus compte que le microbiote
associé à un hôte, que l'on parle de flore commensale ou symbiotique
\footnote{Le terme de symbiote sera utilisé ici au sens lagre, a savoir
comprennant à la fois les symbiotes au sens strict (bénéfice réciproque), mais
aussi les parasites, le point dommun étant le carractère obligatoire de la
relation qu'il entretient avec son hôte.}, a un impact important sur les
traits d'histoire de vie de son hôte.\footnote{TODO : Citer une source,
sachant qu'il y en a plein qui parlent de ça. Peu⁻être une review...} De ce
constat découle la notion d'holobionte\cite{rosenberg2007}, qui consiste à
considérer en tant qu'unité séléctve non plus le génotype d'une espèce, mais
ceux, combinés, de toute sa microflore et de lui-même, formant ainsi une sorte
de méta-génome, et potentiellement de méta-organisme sur lequel la pression de
sélection s'éxercerait.

% La résistance au stress, en particulier thermique, est une carractéristique
% principale des espèces dites invasives, notamment chez les insectes. La
% compréhension des mécanismes sous-jascents de cette résistance présente donc
% un enjeu majeur, tant pour des raisons scientifiques que de santé publique. En
% effet, certaines espèces invasives sont extrêmement problématiques, car
% pouvant être des vecteurs de maladies souvent virales, comme par exemple
% \esp{Aedes albopictus}, vecteur du Chikungunya, de la Dengue et d'un grand
% nombre d'autres virus.

%%%%%%%% 2ème paragraph %%%%%%%%%%%%%%%%%%%%%%%
% 2ème partie : précise l'aspect particulier du problème (VA VERS L'INCONNU)
%%%%%%%%%%%%%%%%%%%%%%%%%%%%%%%%%%%%%%%%

\paragraph{} % (fold)
\label{par:intro1}


% Or, il a été montré\footnote{ref. nécessaire} que les bactéries symbiotiques
% des insectes influent e façon significative sur la biologie de leurs hôtes,
% modifiant leurs capacités d'adaptation, de résistance à divers stress, et même
% dans certains cas leurs carractéristiques reproductives.

%%%%%%%% 3ème paragraph %%%%%%%%%%%%%%%%%%%%%%%%
% 3ème partie : indique le but du travail et un apercu des résultats 
% (POSE LA QUESTION)
%%%%%%%%%%%%%%%%%%%%%%%%%%%%%%%%%%%%%%%%%%

\paragraph{} % (fold)
\label{par:intro3}


% Nous nous intéresserons donc dans ce mémoire au rôle que peut avoir la
% communauté symbiotique des insectes dans leur résistance au stress thermique.
