\documentclass[%
	a4paper,
	11pt,
	linktocpage=true,
	oneside,
	DIV=calc, 
	% DIV règle les proportions corps de texte-marges. Classic, c'est pour
	% suivre les canons du Moyen-âge. Pardonnez-moi, prince.
	% Note : à cause des règles à la con pour le mémoire, je vais devoir passer par-dessus avec geometry…
	%]{scrreprt}
	]{scrartcl}
\usepackage{crapstyle}

% La biblio...
% Le style bibliographique AEM a été pris ici : https://github.com/roey-angel/BIBEMME
% Merci à Roey-Angel pour ce partage.
\usepackage[
	%sorting=nyt,
	%style=reading, % Le style reading est un mode bavard, il imprimme l'abstract, le champs d'entrée dans la BdD, etc.
	style=aem,
	natbib=true, % Ça je sais plus à quoi ça sert, à vérifier.
	backend=biber % Par-ce que j'aime Justin Biber. Non, c'est la version de BibLaTeX qui lit l'UTF-8 ;)
	]{biblatex}
%\renewcommand*{\bibfont}{\small}
\addbibresource{bib.bib} % Bib bob bidou blop !

% \title{Rôle de la communauté symbiotique bactérienne dans la réponse des insectes au stress thermique}
\title{Réponses de l'insecte aux stress thermiques et rôle de la communauté symbiotique bactérienne}

% Cet environnement Figure ne servira que si on est obligé d'insérer une image
% dans un environnement multicol. Mais c'est moche.
\newenvironment{Figure}
  {\par\medskip\noindent\minipage{\linewidth}}
  {\endminipage\par\medskip}

\newenvironment{note}
   {\begin{quote} \small \linespread{1}\textbf{Note temporaire~:~}}
   {\normalsize \end{quote}\linespread{1.5}}

\newcommand{\esp}[1]{\textit{#1}} % Commande pour stocker les noms d'espèces.
\newcommand{\gene}[1]{\textit{#1}} % Commande pour stocker les noms de gènes.
\newcommand{\eng}[1]{\textit{#1}} % Commande pour mettre des trucs en anglais

%Pour éviter les sauts de page à chaque chapitre.
% \makeatletter
% \renewcommand\chapter{%
%                     \thispagestyle{plain}%
%                     \global\@topnum\z@
%                     \@afterindentfalse
%                     \secdef\@chapter\@schapter}
% \makeatother



\begin{document}
\renewcommand{\refname}{R\'ef\'erences bibliographiques}

%	\maketitle
%	\begin{abstract}

Ceci est un document de travail pour la rédaction de mon mémoire
bibliographique de recherche pour mon M2. La version twocolumns n'est
évidement pas la version finale (les gens fixant les règles de présentation
ayant des goûts douteux en terme de typographie), mais la version dédiée au
travail sur le fond (ça fait moins de pages à imprimer, et y'a pas à dire,
c'est plus lisible.). Merci de votre compréhension.

\end{abstract}
\thispagestyle{empty}
	\tableofcontents
	\pagebreak % Pasque sinon, l'intro commence direct après la TOC...
\setcounter{page}{1}
\nocite{zhao2012}

	\section*{Introduction} % (fold)
\label{chap:introduction}
\addcontentsline{toc}{section}{Introduction}
	%%%%%%%% 1er paragraph %%%%%%%%%%%%%%%%%%%%%%%%
% 1ère partie : expose l'aspect général du sujet (CONNU)
%%%%%%%%%%%%%%%%%%%%%%%%%%%%%%%%%%%%%%%%

\paragraph{} % (fold)
\label{par:intro1}

Il est de plus en plus admis dans la communauté scientifique que le microbiote
associé à un hôte, que l'on parle de flore commensale ou symbiotique
\footnote{Le terme de symbiote sera utilisé ici au sens lagre, a savoir
comprennant à la fois les symbiotes au sens strict (bénéfice réciproque), mais
aussi les parasites, le point dommun étant le carractère obligatoire de la
relation qu'il entretient avec son hôte.}, a un impact important sur les
traits d'histoire de vie de son hôte \cite{feldhaar2011}. De ce constat
découle la notion d'holobionte \cite{rosenberg2007}, qui consiste à considérer
en tant qu'unité séléctve non plus le génotype d'une espèce, mais ceux,
combinés, de toute sa microflore et de lui-même, formant ainsi une sorte de
méta-génome, et potentiellement de méta-organisme sur lequel la pression de
sélection s'éxercerait.

Pour prendre l'exemple des insectes, nous pouvons par exemple citer le cas de
la bactérie \esp{Buchnera sp.},  endosymbiotique du puceron, jouant un rôle
essentiel dans le métabolisme de son hôte,  en synthétisant des acides aminés
que le puceron n'arriverait pas à obtenir en quantité suffisante dans la sève
élaborée, qui constitue sa seule source de nourriture \cite{douglas1998}.
\todo{Rédiger exemple aphid color}

% La résistance au stress, en particulier thermique, est une carractéristique
% principale des espèces dites invasives, notamment chez les insectes. La
% compréhension des mécanismes sous-jascents de cette résistance présente donc
% un enjeu majeur, tant pour des raisons scientifiques que de santé publique. En
% effet, certaines espèces invasives sont extrêmement problématiques, car
% pouvant être des vecteurs de maladies souvent virales, comme par exemple
% \esp{Aedes albopictus}, vecteur du Chikungunya, de la Dengue et d'un grand
% nombre d'autres virus.

%%%%%%%% 2ème paragraph %%%%%%%%%%%%%%%%%%%%%%%
% 2ème partie : précise l'aspect particulier du problème (VA VERS L'INCONNU)
%%%%%%%%%%%%%%%%%%%%%%%%%%%%%%%%%%%%%%%%

\paragraph{} % (fold)
\label{par:intro2}

Parmi les insectes, justement, se retrouve bon nombre d'espèces considérées
comme invasives, voir extrêmement invasives comme le moustique tigre
\esp{Aedes albopictus}. Leur carractère invasif est extrêmement lié à une
capacité à s'acclimater à une variation climatique brutale.

% En plus d'être invasifs, ils sont méchants. 

L'intérêt de l'étude du déterminisme de cette capacité d'adaptation est
d'autant plus important que ces insectes invasifs sont parfois vecteurs
maladies, comme c'est le cas d'\esp{Aedes albopictus}, cité précédement, ce
qui fait de l'étude de ces insectes une problématique majeure de santé
publique \cite{schaffner2013}.

% Or, il a été montré\footnote{ref. nécessaire} que les bactéries symbiotiques
% des insectes influent e façon significative sur la biologie de leurs hôtes,
% modifiant leurs capacités d'adaptation, de résistance à divers stress, et même
% dans certains cas leurs carractéristiques reproductives.

%%%%%%%% 3ème paragraph %%%%%%%%%%%%%%%%%%%%%%%%
% 3ème partie : indique le but du travail et un aperçu des résultats 
% (POSE LA QUESTION)
%%%%%%%%%%%%%%%%%%%%%%%%%%%%%%%%%%%%%%%%%%

\paragraph{} % (fold)
\label{par:intro3}

Il serait donc intéressant d'envisager l'étude de cette résistance au stress
thermique sous un nouvel angle, celui qui considère non plus comme unité
évolutive l'insecte seul, mais l'ensemble que forme l'hôte insecte et ses
bactéries symbiotiques et commensales. Autrement dit, nous chercherons à
déterminer l'impact qu'a la µflore des insectes sur leurs capacités de
résistance au stress thermique.

Nous décrirons dans un premier temps les mécanismes de réponse au stress
thermique propres à l'insecte, puis dans un second temps, nous détaillerons
des exemples de cas dans lesquels la communauté microbienne est
particulièrement impliquée dans les phénomènes de résistance au stress
thermique.

\begin{note}
Le plan n'étant pas encore vraiment fixé (à part pour les grandes parties), l'annonce du plan est quelque-peu succint.
\end{note}


% Nous nous intéresserons donc dans ce mémoire au rôle que peut avoir la
% communauté symbiotique des insectes dans leur résistance au stress thermique.

\chapter{Réponses de l'insecte aux stress de température}

% Préambule accilmatation vs. adaptation
En biologie tout stress appelle une réponse de l'organisme pour maintenir l'équilibre biologique dans un état fonctionnel.
Deux types de réponses, souvent confondues à tort, sont observées lors d'un stress : l'adaptation et l'acclimatation.
La différence se situe essentiellement sur l'échelle de temps durant laquelle se déroule le phénomène.
Dans le cas de l'adaptation, les organismes seront confrontés à un changement environnemental sur l'échelle de plusieurs générations.
Ce changement sur le long terme va introduire une pression de séléction qui va à plus ou moins long terme modifier le génome de la lignée (ou la composition de l'holobionte), qui sera plus adaptée à ce nouvel environnement.
Das le cas de l'acclimatation, l'échelle de temps est réduite à celle de la vie d'un individu. La réponse au changement environnemental sera dépendant des capacités intrinsèques de l'individu ou de l'holobionte en question, et ces changements seront réversibles.
On parle aussi de plasticité adaptative de l'individu pour désigner cette capacité d'acclimatation.

	\section{Les protéines de choc thermique, clé de voûte de la réponse acclimative}

	\section{Exemples de réponse chez les insectes}

		\subsection{Induction de lé réponse de <<~Heat-Shock~>>}

		\subsection{Autres types de réponses} % À reformuler.


\chapter{Interférences du microbiote dans la réponse au stress de température} % (fold)
	\label{sec:implicationµbiote}

De manière générale, il est souvent constaté que les conditions stressantes favorisent les relations de mutualisme \cite{meadows2013}.
Nous sommes donc en droit d'émettre l'hypothèse selon laquelle les partenaires mutualistes joueraient un rôle significatif dans la réponse au stress.

La diversité des mécanismes pouvant mettre en relation le microbiote bactérien
et leur hôte dans un phénomène de réaction à un stress thermique est très
grande, et fait surtout l'objet de recherches toujours actives. Nous décrirons
dans cette partie les mécanismes les plus fréquemment retrouvées à l'heure
actuelle.


	\section{Interférences directes} % (fold)
		\label{sec:direct}

\subsection{Stimulation de gènes de l'hôte} % (fold)
\label{sub:genes_hote}

Dans ce type de relation, la présence de la bactérie dans les tissus de
l'insecte stimule l'expression de protéines de stress, quelque soit le
contexte environemental. Les protéines de stress sont donc exprimées de façon
<<artificiellemnt>> constitutive, ce qui permet à l'insecte de résister de façon
plus efficace à un stress brutal et rapide.

\subsection{Exemples} % (fold)
\label{sub:exemples}

% Exemple Rickettsia (brumin2011)

Notre premier exemple concernera le couple symbiotique que forme la mouche
\esp{Bemisia tabaci}, un ravageur de culture et vecteur de virus
phytopathogènes, et son symbiote secondaire \esp{Rickettsia sp.}. Une étude
\cite{brumin2011} a récement mis en évidence un mécanisme conduisant à une
meilleure résistance de la mouche au stress thermique, lorsqu'elle est
associée à sa bactérie symbiotique. Le processus mis en jeu dans cette
résistance est le suivant : la
présence de la bactérie dans les tissus de la mouche induit une expression de
gènes nécessaires à la thermotolérance (notamment ici des gènes codant pour
des éléments du cytosquelette), même dans des insectes non-soumis à un stress
thermique.
Plus spécifiquement, une autre étude \cite{tang2012} montre une induction de HSP70 par l'introduction de protéines provenant de cultures d'\esp{Escherichia coli} et \esp{Staphilococcus aureus} dans la larve de la mouche domestique (\esp{Musca domestica}).

Il pourrait y avoir deux types d'interprétation de ces résultats.
Nous savions déjà que l'expression de HSP est une réponse plutôt générique à un stress cellulaire.
L'infection bactérienne provoquerait-elle un stress (évolutivement<<~mis en place~>> pour la protection contre les patogènes), dont la réponse générique accidentellement exploitée en réponse à un choc thermique ?
Ou bien cette réponse est une adaptation spécifique de l'holobionte afin de répondre aux stress (génériques ou spécifiques) ?

	\section{Interférences indirectes} % (fold)
	 	\label{sec:indirect}

\subsection{La réponse au stress du microbiote affecte la réponse de l'hôte} % (fold)
\label{sub:groel}

% Exemple Bushnera et mutation gène ibpA (dunbar2007)

Notre premier exemple portera sur l'association entre \esp{Bushnera sp.} et le
puceron \esp{Acyrthosiphon pisum}. Une étude \cite{dunbar2007} a montré que la mutation d'une seule nucléotide dans le promoteur d'une petite HSP de \esp{Bushnera} induisait
chez l'hôte du mutant une résistance moindre aux choc thermiques.

Le cas de l'expression d'une protéine de choc thermique GroEL par les
organelles issues d'une ancienne endosymbiose (voir partie
%\ref{ssub:la_famile_hsp60}
1) très similaire à celle produite par les bactéries
suggère la possibilité d'une utilisation de l'expression bactérienne de
protéines de réponse au stress, par leur hôte. 
On observe effectivement ce type de mécanisme dans l'étude \cite{stoll2009}.

% Exemple GroEL

Dans ce deuxième exemple, la protéine GroEL est sur-exprimée par
l'endosymbionte \esp{Blochmannia floridanus} en cas de stress, et qui diffuse
dans l'hémolymphe de son hôte, la fourmi \esp{Campontus floridanus}.
Une fois dans l'hémolymphe, les protéines de stress bactériennes peuvent
exercer leurs fonctions de chaperonnes sur les protéines dénaturées de l'hôte. 
% \todo[inline]{Peut-être trouver un autre exemple, celui-ci n'est
% pas clairement le propos de la publi}




\subsection{Dynamique de densité des populations symbiotiques} % (fold)
\label{sub:taillepop}

Les contitions climatiques sont souvent corrélées à un changement parfois
radical dans la nature de la population bactérienne associée à l'hôte. Il a
par exemple été montré \cite{harmon2009} que les populations de pucerons
issues de milieux soumis à des chocs thermiques réguliers présentaient une
prévalence de leurs symbiontes secondaires bien plus élevés que ceux de pays
plus cléments. 
Il s'agit là d'une adaptation.
En revanche, il a été de nombreuses fois démontré qu'un choc thermique brutal, qu'il soit chaud ou froid, diminuait la densité de l'endosymbiote.
Celui-ci étant associé de très près à des fonctions vitales de l'insecte, en particulier dans la nutrition, des changements dans sa densité affecterait très fortement la fitness de l'hôte, et donc interférerait avec sa raponse au stress.

De plus, il n'est pas rare qu'au sein d'un même hôte plusieurs populations de symbiotes
cohabitent.
% Pour reprendre le cri d'amour du crapaud.
Il se met alors en place des intéractions classiques entre ces
populations, pouvant aller de l'antagonisme au commensalisme. Par exemple, il
a été montré \cite{montllor2002} que la présence de symbiotes secondaires
comme \esp{Rickettsia sp.} protégeait les bactériocytes\footnote{Les
bactériocytes sont des cellules de l'hôte spécialisées dans la relation avec
son endosymbiote, dans le cas du puceron, \esp{Bushnera sp.}} du puceron,
lorsque celui-ci est soumis à un choc thermique.

Une autre étude \cite{bordenstein2011} concernant la relation tripartite entre l'hôte insecte, l'endosymbionte \esp{Wolbachia} et le bactériophage WO montre que le stress thermique a non seulement un effet sur la densité de \esp{Wolbachia} l'inductuin de la phase lytique de WO, mais aussi sur le seuil de densité a partir duquel \esp{Wolbachia} induit l'incompatibilité cytoplasmique (IC)%
\footnote{Définir IC} sur son hôte.
% Objection : Ici c'est pas une interférence dans la réponse...
Sachant que l'IC impacte grandement la fitness de l'hôte insecte, sa modification par un stress thermique modifie significativement la réponse de l'holobionte à ce stress.

% \subsection{diminution antagonismes} % (fold)
% \label{sub:diminution_antagonismes}

% → \esp{Wolbachia} et diminution de l'IC pendant ub HS.
% \todo[inline]{Essayer de trouver le fulltext de \citet{arakaki2001}, qui parle de ça. Sinon, on vire la partie (ou on trouve une autre publi là-dessus)}

\section*{\textsc{Conclusion}}
\addcontentsline{toc}{section}{\textsc{Conclusion}}

La réponse aux stress thermiques constitue un processus complexe, de part la multiplicité des mécanismes mis en jeu. 
Certaines de ces réponses, comme par exemple, les réponses éthologiques (fuite, migrations, construction d'abris\ldots)  n'ont pas été évoquées dans ce mémoire et constituent pourtant une réaction non négligeable de l'insecte face à un stress de ce type.
De plus, nous n'avons évoqué dans ce mémoire uniquement la réponse de l'insecte au stress chaud, en négligeant volontairement les aspects liés au stress froid, pourtant primordial dans la biologie des insectes soumis aux fluctuations environnementales.
% Nous n'avons pas, par exemple, évoqué les réponses éthologiques (fuite, migrations, construction d'abris...), et sans doute passé sous silence certains éléments physiologiques.
Jusqu'à présent, l'étude de la réponse au stress thermique a souvent été considérée à l'échelle même de l'individu ou des populations.
Avec la prise en compte récente du concept d'holobionte, l'organisme n'est plus considéré comme un individu isolé mais comme une communauté d'espèces (lui-même et tous les organismes qu'il renferme).
% Cela reflète la grande diversité des réponses à ces stress, et le fait que de part son importance dans les écosystèmes vivants, ces différentes réponses ont déjà été étudiées sous beaucoup d'angles.
Cette nouvelle dimension permet aujourd'hui d'appréhender sous un nouvel angle, plus intégratif, les processus adaptatifs des insectes aux changements de conditions climatiques.
Or ces derniers jouent un rôle important dans les processus de migration observés actuellement chez les moustiques vecteurs, notamment vers les pays aux climats plus froids que leurs régions d'origine.
Un exemple-type serait celui du moustique tigre, \esp{Aedes albopictus}, originaire d'Asie du Sud-Est, colonisant peu à peu nos régions tempérées, et dont la tolérance au froid pourrait être un des points clés pour expliquer son adaptation.
De façon intéressante, il a été montré que ces moustiques hébergent un microbiote bactérien riche et varié dont le rôle dans la thermotolérance reste encore à déterminer.
% Il n'a cependant pas été beaucoup étudié sous l'angle de l'holobionte, notion qui permet pourtant d'ajouter un grand facteur de plasticité dans notre compréhesion de l'interface individu-environement.
% Cependant la prise de conscience de cette dimention est aujourd'hui en plein essor, notamment en raison de l'importance grandissante des épidémies d'arboviroses.

C'est justement dans ce contexte que mon stage de master se déroulera, il consistera à mesurer les interférences entre la réponse du moustique tigre \esp{Aedes albopictus} au stress froid et l'un de ses symbiotes secondaires cultivables, la bactérie \esp{Acinetobacter calcoaecitus}.


%\chapter{Implication du microbiote dans la réponse au stress de température} % (fold)
\label{sec:implicationµbiote}
	
	\section{Exemples clés} % (fold)
	\label{sec:exemples}
		Nous allons maintenant détailler quelques exemples de cas dans lesquels le
partenaire microbien a un rôle prépondérent dans les mécanismes de réaction au
stress thermique de leur hôte. 


		\subsection{\esp{Rickettsia sp.} et les aleurodes du tabac (\textit{whiteflies})} % (fold)
		\label{sub:rickettsia_et_les_aleurodes}
			\paragraph{description} % (fold)
\label{par:description_whitefly}
\cite{brumin2011}

\paragraph{Processus mis en jeu} % (fold)
\label{par:process_whitefly}


			% Via feldhaar2011

		\subsection{\esp{Buchnera sp.} et le puceron} % (fold)
		\label{sub:exemple_buchnera}
			\paragraph{Description} % (fold)
\label{par:buchnera_description}


\cite{dunbar2007}

\paragraph{Processus mis en jeu} % (fold)
\label{par:buchnera_process}

La bactérie sur-exprime la protéine chaperonne groEL (protéine de stress
thermique bactérienne), qui diffuse dans l'hémolymphe de l'hôte. Une fois dans l'hémolymphe, groEL peut exercer sa fonction chaperonne sur les protéines insectes, ce qui confère à l'hôte une aire pour résister au choc thermique.


		\subsection{Rôle de groEL dans l'hôte \esp{Nezara viridula}} % (fold)
		\label{sub:groel}
			\paragraph{Description} % (fold)
\label{par:description_groel}

% \cite{wilcox2003}
 \cite{stoll2009}

\paragraph{mécanismes mis en jeu} % (fold)
\label{par:process_groel}


	\section{Diversité des mécanismes} % (fold)
	\label{sec:diversit_des_m_canismes}
		\input{content/part2/diversité_mécanismes}

		\subsection{Stimulation de l'expression de gènes de l'hôte} % (fold)
		\label{sub:gènes_hôte}
			Exemple : partie \ref{sub:rickettsia_et_les_aleurodes} et \ref{sub:exemple_buchnera}

		\subsection{Utilisation par l'hôte de protéines microbiennes} % (fold)
		\label{sub:gènes_microbiens}
			\input{content/part2/gènes_bactos}

%\chapter*{Perspectives}
\addcontentsline{toc}{chapter}{Perspectives}

\todo[inline]{Limite entre acclimatation et adatation pas sur la même échelle de temps entre le microbiote et la bestiole, ce qui fait que quand l'insecte ne peut pas s'adapter (de façon héritable), le microbiote, ayant fait plusieurs générations, a eu le temps d'ajuster son génome par séléction (entre autres).}

\todo[inline]{Pas sûr de mettre cette partie, j'ai déjà trop de pages, et la rédaction de certaines parties n'est pas encore terminée...}

%\linespread{1}
%\newgeometry{bottom=2cm}
\pagebreak
\sloppy
\printbibliography%[sorting=nty]
\addcontentsline{toc}{section}{\textsc{Références bibliographiques}}
% \listoftodos

\end{document}
