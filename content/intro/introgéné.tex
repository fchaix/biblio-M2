%%%%%%%% 1er paragraph %%%%%%%%%%%%%%%%%%%%%%%%
% 1ère partie : expose l'aspect général du sujet (CONNU)
%%%%%%%%%%%%%%%%%%%%%%%%%%%%%%%%%%%%%%%%

\paragraph{} % (fold)
\label{par:intro1}

La communauté scientifique se rend de plus en plus compte que le microbiote
associé à un hôte, que l'on parle de flore commensale ou symbiotique
\footnote{Le terme de symbiote sera utilisé ici au sens lagre, a savoir
comprennant à la fois les symbiotes au sens strict (bénéfice réciproque), mais
aussi les parasites, le point dommun étant le carractère obligatoire de la
relation qu'il entretient avec son hôte.}, a un impact important sur les
traits d'histoire de vie de son hôte.\footnote{TODO : Citer une source,
sachant qu'il y en a plein qui parlent de ça. Peu⁻être une review...} De ce
constat découle la notion d'holobionte \cite{rosenberg2007}, qui consiste à
considérer en tant qu'unité séléctve non plus le génotype d'une espèce, mais
ceux, combinés, de toute sa microflore et de lui-même, formant ainsi une sorte
de méta-génome, et potentiellement de méta-organisme sur lequel la pression de
sélection s'éxercerait.

Pour prendre l'exemple des insectes, nous pouvons par exemple citer le cas de
la bactérie \esp{Buchnera sp.},  endosymbiotique du puceron, jouant un rôle
essentiel dans le métabolisme de son hôte, lui procurant des acides aminés que
le puceron n'arriverait pas à obtenir en quantité suffisante dans la sève
élaborée, qui constitue sa seule source de nourriture \cite{douglas1998}.

[TODO : étoffer avec un autre exemple]

% La résistance au stress, en particulier thermique, est une carractéristique
% principale des espèces dites invasives, notamment chez les insectes. La
% compréhension des mécanismes sous-jascents de cette résistance présente donc
% un enjeu majeur, tant pour des raisons scientifiques que de santé publique. En
% effet, certaines espèces invasives sont extrêmement problématiques, car
% pouvant être des vecteurs de maladies souvent virales, comme par exemple
% \esp{Aedes albopictus}, vecteur du Chikungunya, de la Dengue et d'un grand
% nombre d'autres virus.

%%%%%%%% 2ème paragraph %%%%%%%%%%%%%%%%%%%%%%%
% 2ème partie : précise l'aspect particulier du problème (VA VERS L'INCONNU)
%%%%%%%%%%%%%%%%%%%%%%%%%%%%%%%%%%%%%%%%

\paragraph{} % (fold)
\label{par:intro1}

Parmi les insectes, justement, se retrouve bon nombre d'espèces considérées
comme invasives, voir extrêmement invasives comme le moustique tigre
\esp{Aedes albopictus}. Leur carractère invasif est extrêmement lié à une
capacité à s'acclimater à une variation climatique brutale.

% En plus d'être invasifs, ils sont méchants. 

L'intérêt de l'étude du déterminisme de cette capacité d'adaptation est
d'autant plus important que ces insectes invasifs sont parfois vecteurs
maladies, comme c'est le cas d'\esp{Aedes albopictus}, cité précédement, ce
qui font de l'étude de ces insectes une problématique majeure de santé
publique \cite{schaffner2013}.

% Or, il a été montré\footnote{ref. nécessaire} que les bactéries symbiotiques
% des insectes influent e façon significative sur la biologie de leurs hôtes,
% modifiant leurs capacités d'adaptation, de résistance à divers stress, et même
% dans certains cas leurs carractéristiques reproductives.

%%%%%%%% 3ème paragraph %%%%%%%%%%%%%%%%%%%%%%%%
% 3ème partie : indique le but du travail et un apercu des résultats 
% (POSE LA QUESTION)
%%%%%%%%%%%%%%%%%%%%%%%%%%%%%%%%%%%%%%%%%%

\paragraph{} % (fold)
\label{par:intro3}


% Nous nous intéresserons donc dans ce mémoire au rôle que peut avoir la
% communauté symbiotique des insectes dans leur résistance au stress thermique.
