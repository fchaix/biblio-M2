\section{Interférences du microbiote bactérien de l'insecte dans la réponse aux stress de température} % (fold)

\paragraph*{}
% Nous avons vu différentes voies de réponse de l'insecte aux stress thermiques, en réduisant le système à l'insecte seul.
% Or nous savons que dans la nature, la sélection s'exerce sur l'ensemble formé par l'hôte et son microbiote, il s'agit de la notion d'holobionte, défini précédemment.
% Dans un système en constante intéraction entre les partenaires, il est intuitif de supposer que le microbiote puisse jouer un rôle, plus ou moins directement, dans la réponse au stress thermique, de laquelle dépend la fitness de l'holobionte ?
De manière générale, il est souvent constaté que les conditions stressantes favorisent les relations de mutualisme \cite{meadows2013}.
Il est d'ailleurs admis que dans la nature, la sélection s'exerce sur l'ensemble formé par l'hôte et son microbiote ou holobionte \cite{rosenberg2008}.
Dans ce contexte, des recherches actives sont actuellement menées pour mieux comprendre les mécanismes mis en jeu dans l'interaction hôte-microorganismes, en réponse à un stress thermique.
Des exemples actuels décrivent l'existence d'interférences microbiennes plus ou moins directes avec la réponse de l'insecte aux stress thermiques

La diversité des mécanismes pouvant mettre en relation le microbiote bactérien et leur hôte dans un phénomène de réaction à un stress thermique est très grande, c'est pourquoi nous décrirons dans cette partie les mécanismes les plus fréquemment retrouvés à l'heure actuelle, résumés dans le tableau~2.

\subsection{Stimulation de l'expression de gènes de l'hôte}

L'infection bactérienne est connue comme étant un déclencheur d'une réponse heat shock (Figure~1) \cite{sorensen2003}.
Il est donc naturel que la présence d'une bactérie endosymbiotique dans les tissus de l'insecte stimule l'expression de protéines de stress, quelque soit le contexte environnemental.
Les protéines de stress sont donc exprimées de façon <<~artificiellement~>> constitutive, ce qui permet à l'insecte d'être mieux armé et ainsi de mieux faire face à un stress brutal et rapide.

% Dans le premier type de relation que nous allons citer, la présence de la bactérie au contact des tissus de l'insecte stimule l'expression de protéines de stress, ou du moins de thermotolérance, quelque soit le contexte environnemental.
% En effet, l'infection bactérienne est connue comme étant un déclencheur d'une réponse de heat shock (voir tableau 1) \cite{deitch1995}.
% L'hôte est donc dans un état de stress <<~artificiel~>>, y compris dans un environnement non stressant, ce qui lui permet d'être <<~préparé~>> à un éventuel choc thermique, et donc de mieux y résister.

% Notre premier exemple concernera le couple symbiotique que forme la mouche \esp{Bemisia tabaci}, un ravageur de culture et vecteur de virus phytopathogènes, et son symbiote secondaire \esp{Rickettsia sp.} \cite{brumin2011}.
% Cette étude a mis en évidence un mécanisme conduisant à une meilleure résistance de la mouche au stress thermique, lorsqu'elle est associée à \esp{Rickettsia sp.}.
% Partant du constat que les populations associées à \esp{Rickettsia sp.} étaient plus résistants aux chocs thermiques chauds (incubation des femelles adultes à différentes températures, pendant 3h30), les chercheurs ont mesuré l'expression de gènes connus comme faisant partie de la réponse thermotolérante.
% Les résultats montrent que les HSPs, contrairement à ce à quoi l'on pourrait s'attendre, ne sont pas significativement induits par la présence de \esp{Rickettsia sp.}, mais qu'en revanche diverses protéines associées au cytosquelette sont nettement induites.
% Or, on sait que le cytosquelette, de part sa structure protéique, est l'un des organes cellulaire le plus rapidement touché par les chocs thermiques.
% \esp{Rickettsia sp.} semblerait capable d'induire la sur-expression de protéines du cytosquelette de leur hôte, induisant ainsi une meilleure résistance aux stress thermiques.

Un des exemples connus concerne l'association symbiotique entre la mouche \esp{Bemisia tabaci}, un ravageur de culture et vecteur de virus phytopathogènes, et son symbiote secondaire \esp{Rickettsia sp.}.
Il a en effet été montré qu'en présence de la bactérie, la mouche présentait une meilleure résistance à un stress thermique de 40\textdegree{}C, comparé aux insectes dépourvus de ces symbiotes \cite{brumin2011}.
%Une hypothèse avancée par les auteurs serait que a présence de la bactérie dans les tissus de la mouche induirait l’expression de gènes impliqués dans la thermotolérance.
Contrairement aux résultats attendus, les HSPs n'ont pas été significativement induites par la présence de \esp{Rickettsia sp.}.
En revanche, les résultats ont montré un très nette augmentation de la synthèse diverses protéines associées au cytosquelette.
Or, il est connu que le cytosquelette, de part sa nature protéique, est l'une des structures cellulaire les plus rapidement impactées suite à un choc thermique \cite{brumin2011}.
Il semblerait donc que \esp{Rickettsia sp.} induirait la sur-expression de protéines du cytosquelette de son hôte, permettant ainsi une meilleure résistance de ce dernier aux stress thermiques.

Un dernier exemple concerne un modèle insecte très étudié dans le cadre de ses symbioses, le puceron.
Outre son endosymbiote obligatoire \esp{Bushnera sp.} connu pour fournir à son hôte des compléments essentiels à son alimentation carencée en certains acides aminés, le puceron \esp{Acyrthosiphon pisum} héberge parfois un symbiote facultatif, \esp{Serratia symbiotica}, connu pour conférer à son hôte une plus grande thermotolérance\footnote{La thermotolérance désigne la capacité d'un individu, ou d'un holobionte, à supporter de grands écarts de température. Autrement dit, la capacité de l'individu ou de l'holobionte à s'acclimater avec succès à des températures différant de leur optimum physiologique.} \cite{montllor2002}.
Les mécanismes sous-jascents de cette termotolérance demeurent aujourd'hui inconnus.
Cependant, une étude récente de Burke et Moran apporte des éléments de réponse à cette question \cite{burke2011}.
Un profilage du transcriptome de pucerons infectés et non infectés par \esp{Serratia symbiotica} a été effectué, afin de déterminer la nature et la quantité des gènes induits par l'infection du puceron par le symbiote.
Les résultats montrent une faible différence entre les deux transcriptomes, à la fois d'un point de vue qualitatif (28 gènes) que quantitatif.
Or, une étude précédement effectuée par la même équipe a confirmé une grande différence d'un point de vue métabolique entre les deux conditions \cite{burke2009}.
Les auteurs interprètent cette faible réponse du transcriptome à l'infection, contrastant cette termotolérance et cette réponse métabolique importante, en supposant que l'impact de la symbiose se situerait à un niveau post-transcriptionel, ou qu'il serait simplement le reflet de la seule activité métabolique du symbiote.
La réponse au stress du microbiote affecterait dans ce cas la réponse de l'hôte.

% Tang 2012 → Je suis un con, j'avais mal compris la publi, c'était pas ça.

\subsection{La réponse au stress du microbiote affecte la réponse de l'hôte}

Le microbiote peut, comme son hôte, être directement affecté par les variations de température, mettre en place des réponses à ces chocs afin d'y résister.
Il a été montré que les molécules bactériennes produites en réponse au stress thermique peuvent aussi, en raison de la grande proximité spatiale hôte-microbiote, profiter à l'hôte.

Une étude menée sur l'association entre la bactérie endosymbiotique \esp{Bushnera sp.} et le puceron \esp{Acyrthosiphon pisum} a montré que la mutation d'un seule nucléotide dans le promoteur du gène \gene{ibpA} codant pour une petite HSP de la bactérie induisait chez l'hôte une résistance moindre aux choc thermiques, en changeant la température stimulant son expression \cite{dunbar2007}.
% Notre premier exemple portera sur l'association entre \esp{Bushnera sp.} et le puceron \esp{Acyrthosiphon pisum}.
% Une étude a montré qu'une mutation, existante dans la nature, d'une seule nucléotide dans le promoteur d'une petite HSP de \esp{Bushnera} induisait chez l'hôte du mutant une résistance moindre aux choc thermiques \cite{dunbar2007}.
% Cette mutation dans le promoteur de \gene{ibpA}, un gène codant pour une petite HSP, change le contexte dans lequel l'expression de ce gène va se produire.
En effet, dans la souche sauvage, \gene{ibpA} s'exprime en réponse à un choc thermique chaud, alors que chez les mutants, \gene{ibpA} ne s'exprime qu'en réponse à des températures froides.
Les résultats ont ainsi montré que la population de pucerons infectés par la souche sauvage avait un meilleur taux de survie dans un contexte de choc thermique chaud, mais que l'inverse était vrai aussi : les pucerons qui portent les bactéries possédant l'allèle muté sont plus aptes à survivre dans un environement froid.

% Exemple GroEL
Un autre exemple concerne les familles de protéines hsp60/hsp10, hautement conservées dans le vivant puisque leur équivalent est aussi bien retrouvé dans le génome eucaryote (le complexe Tcp-1), que dans celui des bactéries (GroEL/GroES) ou encore des organelles des eucaryotes \cite{gupta1995}.
Cette similitude est telle que l'hypothèse d'une origine commune semble confortée, et constitue même un des fondements de la théorie endosymbiotique des organelles chez les eucaryotes \cite{gupta1995}.
En situation de stress, il a ainsi été montré que l'endosymbionte Blochmannia floridanus surexprimait la protéine GroEL qui diffuserait ensuite dans l’hémolymphe de son hôte, la fourmi \esp{Campontus floridanus} \cite{stoll2009, feldhaar2011}.
% Dans ce deuxième exemple, il est montré que la protéine GroEL est sur-exprimée par l'endosymbionte \esp{Blochmannia floridanus} en cas de stress, et qu'elle diffuse dans la cellule de son hôte, la fourmi \esp{Campontus floridanus} \cite{stoll2009}.
L'endosymbionte jouerait ici un rôle similaire à celui des mitochondries de l'hôte.
Une fois circulant dans les cellules de l'hôte et dans l'hémolymphe, de la même façon que le feraient les GroEL produites par les organelles, la chaperonne exercerait ses fonctions de renaturation sur les protéines de l'hôte.

\subsection{Rôle de la dynamique de densité des populations symbiotiques}

Les conditions climatiques sont souvent corrélées à une mutation profonde des communautés de la population bactérienne associée à un hôte, \textit{a fortiori} lorsque l'hôte en question n'est pas un homéotherme.
Il a par exemple été montré que les populations de pucerons vivant dans des milieux soumis à des fluctuations régulières de température présentaient une prévalence de leurs symbiotes secondaires bien plus élevée que ceux issus de pays aux climats plus tempérés \cite{harmon2009}.

De même, Feder \textit{et coll.} (1999) ont émis l'hypothèse d'un lien direct entre le stress thermique régulant la densité de la communauté des bactéries \esp{Wolbachia} chez la drosophile et la capacité de ces endosymbiontes d'induire l'incompatibilité cytoplasmique%
\footnote{L'incompatibilité cytoplasmique est un des effets de \esp{Wolbachia} sur son hôte. Dans les populations infectées, le croisement entre des mâles non infectés et des femelles infectées sont très peut viables, et même parfois impossibles. \esp{Wolbachia} se transmettant de mère à descendants, cette manipulation de la sexualité de l'hôte favorise donc l'expention de la bactérie.}%
, modifiant ainsi la fitness de l'insecte et donc de l'holobionte \cite{feder1999}. 

Une autre étude concernant la relation tripartite entre la guêpe \esp{Nasonia vitripennis}, l'endosymbionte \esp{Wolbachia} et le bactériophage WO a montré que le stress thermique avait non seulement un effet sur la densité de \esp{Wolbachia}, mais aussi l'induction de la phase lytique de WO, et le seuil de densité a partir duquel la bactérie \esp{Wolbachia} induit l'incompatibilité cytoplasmique chez la guêpe \cite{bordenstein2011}.
%Des populations de la guêpe ont été élevées sous des températures différentes (18°C, 25°C et 32°C). La densité de Wolbachia a été mesurée par RT-qPCR
% Plus taaaaard !!!
