
\section*{Conclusion}
\addcontentsline{toc}{section}{Conclusion}

La réponse aux stress thermiques constitue un processus complexe, de part la multiplicité des mécanismes mis en jeu. 
Certaines de ces réponses, comme par exemple, les réponses éthologiques (fuite, migrations, construction d'abris\ldots)  n'ont pas été évoquées dans ce mémoire et constituent pourtant également une réaction de l'insecte face à un stress de ce type.
% Nous n'avons pas, par exemple, évoqué les réponses éthologiques (fuite, migrations, construction d'abris...), et sans doute passé sous silence certains éléments physiologiques.
Jusqu'à présent, l'étude de la réponse au stress thermique a souvent été considérée à l'échelle même de l'individu ou des populations.
Avec la prise en compte récente du concept d'holobionte, l'organisme n'est plus considéré comme un individu isolé mais comme une communauté d'espèces (lui-même et tous les organismes qu'il renferme).
% Cela reflète la grande diversité des réponses à ces stress, et le fait que de part son importance dans les écosystèmes vivants, ces différentes réponses ont déjà été étudiées sous beaucoup d'angles.
Cette nouvelle dimension permet aujourd'hui d'appréhender sous un nouvel angle, plus intégratif, les processus adaptatifs des insectes aux changements de conditions climatiques.
Or il s'avère que ces derniers jouent un rôle important dans les processus de migration observés actuellement chez les moustiques vecteurs.
De façon intéressante, ces moustiques hébergent des microorganismes dont le rôle dans la thermotolérance reste encore à déterminer. 
% Il n'a cependant pas été beaucoup étudié sous l'angle de l'holobionte, notion qui permet pourtant d'ajouter un grand facteur de plasticité dans notre compréhesion de l'interface individu-environement.
% Cependant la prise de conscience de cette dimention est aujourd'hui en plein essor, notamment en raison de l'importance grandissante des épidémies d'arboviroses.

C'est justement dans ce contexte que mon stage de master se déroulera, il consistera à mesurer les interférences entre la réponse du moustique tigre \esp{Aedes albopictus} au stress froid et l'un de ses symbiotes secondaires cultivables, la bactérie \esp{Acinetobacter calcoaecitus}.

