\section{Réponses de l'insecte aux stress de température}

% Préambule acclimatation vs. adaptation
En biologie tout stress appelle une réponse de l'organisme pour maintenir l'équilibre biologique dans un état fonctionnel.
Deux types de réponses, souvent confondues à tort, sont observés lors d'un stress : l'adaptation et l'acclimatation \cite{somero2010}.
La différence se situe essentiellement sur l'échelle de temps durant laquelle se déroule le phénomène.
Dans le cas de l'adaptation, les organismes sont confrontés à un changement environnemental sur l'échelle de plusieurs générations.
Ces modifications du milieu vont alors introduire une pression de sélection qui à terme devra modifier le génome de la lignée (ou celui de l'holobionte par extension), qui sera de ce fait plus adaptée à ce nouvel environnement.
Dans le cas de l'acclimatation, l'échelle de temps est réduite à celle de la vie d'un individu. La réponse au changement environnemental est alors dépendant des capacités intrinsèques de l'individu ou de l'holobionte, et cette réponse sera réversible.
Cette cpacité d'acclimatation de l'individu est souvent qualifiée de plasticité adaptative.
Ce mémoire sera plus particulièrement consacré à l'étude des
phénomènes d'acclimatation suite à un stress ponctuel dans le temps, exercé à l'échelle des individus.

	\subsection{Les protéines de choc thermique, clé de voûte de la réponse acclimative}

	La famille de protéines de réponse au stress évoquée dans cette partie ne concerne pas uniquement les insectes.
	Les HSP (\eng{Heat Shock Proteins}) constituent en effet la réponse la plus universelle au stress thermique qu'il est possible de trouver, des bactéries aux vertébrés supérieurs \cite{lindquist1986}.
	Elles sont de loin les plus documentées dans la littérature.
	Leur découverte remonte à 1962 où Ferruccio Ritossa les a mis en évidence par électrophorèse dans des extraits tissulaires de drosophiles ayant subi un stress chaud \cite{ritossa1996, ritossa1962} %\cite{tissieres1974}.
	Par la suite, elles se sont avérées jouer un rôle bien plus large dans la réponse générique aux stress de diverses natures, comme les radiations aux UV, l'hypoxie, l'exposition aux agents chimiques ou encore les infections (tableau 1) \cite{sorensen2003}.

	Ces protéines appartiennent pour la plupart au grand groupe des protéines chaperonnes \cite{federhoffmann1999}.
	Ces dernières jouent des rôles multiples, en particulier celui de la gestion de la conformation des protéines, leur appariement dans des complexes protéiques, ou encore dans le transport.
	Leur rôle est crucial dans le processus de formation de protéines, y compris dans des conditions environnementales non stressantes.
	Quand survient un stress le besoin accru de réparation rapide des structures endommagées est nécessaire, c'est donc en réponse à ce besoin qu'interviennent les HSPs chaperonnes.
	Le signal précurseur de leur surexpression est le déséquilibre de l'homéostasie des protéines dénaturées \cite{ananthan1986}.
	%De même, il est connu depuis les années 80 que le signal initiateur de la sur-expression de ces protéines est le déséquilibre de l'homéostasie des protéines dénaturées \cite{ananthan1986}.
	Il est donc couramment admis que l'augmentation du nombre de chaperonnes dans un organisme est un indicateur de stress cellulaire \cite{ryan1996}.
	À l'échelle d'observation du vivant, ces protéines sont remarquablement conservées aussi bien dans leur structure que dans certaines de leurs fonctions.
	%Cet indicateur est d'autant plus intéressant pour détecter un stress cellulaire que le taux de conservation dans le vivant de certaines HSPs est important.
	En effet, on retrouve des homologies de séquence entre certaines HSPs, à l'échelle de l'ensemble du vivant, d'où la qualification de leurs gènes respectifs de <<~gènes de ménage~>>, et leur utilisation dans des études phylogénétiques concernant l'ensemble du vivant \cite{gupta1995}.

	Les HSPs sont classiquement dénommées et classées dans de grandes familles en fonction de leur poids moléculaire (par exemple, les protéines de la famille de Hsp70 ont un poids moléculaire d'environ 70\,kDa).
	%, mais de récentes tentatives de clarification de cette nomenclature ont abouti à la proposition de nouvelles dénominations pour ces protéines, sans allusion au poids moléculaire \cite{kampinga2009}.


	\subsection{Exemples de réponse chez les insectes}

		\subsubsection{Induction de la réponse de <<~Heat-Shock~>>}
		% ↑ Peut-être titre à changer, on parle aussi pas mal des moyens d'étude...
		\paragraph*{}
		% De nombreux exemples d'études mettant en évidence la réponse <<~Heat-Shock~>> chez l'insecte peuvent être citées.
		% Nous en sélectionnerons quelque-unes pour tenter de refléter au mieux la diversité des moyens d'étude et des modèles étudiés.

		% D'abord, le stress chaud.

		% La manip historique : Tessière (J'a pas la publi :( )
		%La première mise en évidence de la réponse au heat-shock par des protéines spécifiques a été réalisée sur des échantillons tissulaires du modèle insecte classique : \esp{Drosophila melanogaster}. Des stress thermiques sont été soumis à des populations
		Dans la littérature, il existe de nombreux exemples illustrant la réponse <<~Heat-Shock~>> chez les insectes. Cette dernière diffère principalement selon les modalités du stress thermique opéré, les moyens d'étude,  ou encore les modèles d'insectes étudiés. Un comparatif de ces différentes études est proposé dans le Tableau 2.
		Compte tenu de ses caractéristiques, c'est essentiellement la mouche du vinaigre, \esp{Drosophila melanogaster} qui a fait l'objet du plus grand nombre d'études.
		% De nombreuses études ont bien entendu été réalisées sur la drosophile, modèle classique dans les études sur les insectes.
		Nous essaierons cependant de citer des études sur des modèles variés, ainsi que des types de chocs thermiques différents, afin d'illustrer l'universalité de ce type de réponse.
		Depuis l'étude historique ayant mis en évidence la réponse Heat Shock lors d'un choc thermique \cite{ritossa1996}, par simple observation de variations de condensation de la chromatine, puis de celle qui a mis en évidence les HSPs par éléctrophorèse \cite{tissieres1974}, 
		%des approches quantitatives, moléculaires et enfin protéomiques ont été mises en \oe{}uvre afin de décrire cette réponse heat-shock.
		des approches radicalement différentes ont été mises en \oe{}uvre afin de décrire la réponse HSP chez les insectes.

		% Feder 1997
		Par exemple, une étude a été menée pour étudier in situ la réponse de la larve de la drosophile à un stress thermique induit par une exposition au solein des fruits colonisés par ses larves \cite{feder1997}.
		Les réponses des larves ont été mesurées à différents taux d'ensoleillement, quantifiés grâce à des stations météo.
		Le degré de transcription de la protéine hsp70 utilisée ici comme bioindicateur de la réponse HSP a été mesuré par une méthode immuno-enzymatique ELISA.
		Les résultats de cette expérience ont montré plus de 48\,\% de l'expression de hsp70 à 38\textdegree{}C par rapport au niveau basal de l'expression de la protéine mesurée dans des conditions physiologiques, à 25\textdegree{}C. 

		Une autre étude menée sur la guêpe \esp{Aphidius colemani} a comparé la réponse protéomique des stades nymphaux confrontés à différentes conditions climatiques : un environnement constant et froid, ou bien une alternance de périodes froides et de courts événements chauds \cite{colinet2007}.
%		Pour chaque condition, une extraction des protéines puis une séparation par électrophorèse bidimensionnelle ont été effectuées.
		L'analyse protéomique par spectrométrie de masse a  permis de révéler une surexpression de nombreuses protéines, caractéristiques de la réponse de l'insecte aux conditions variables de température. Parmi elles, figurent les hsp 70 et hsp 90,
		%Les spots différents du contrôle (resté à température ambiante) sont caractérisés par spectrométrie de masse, puis comparée à des bases de données.
		%De nombreuses protéines ont été ainsi mises en évidence comme étant surexprimées dans la condition à température variable, parmi lesquelles figurent des HSPs (hsp70, hsp90) 
		ainsi que des protéines impliquées dans le métabolisme énergétique.

		% Cinétique réponse HS → Nguyen et al 2009

		% Zhao2009 : Identification des gènes exprimés diférement lors d'un HS
		Un autre exemple de réponse au stress thermique chaud concerne le moustique \esp{Aedes aegypti}, vecteur principal de la Dengue et de la fièvre jaune \cite{zhao2009}. Zhao et coll. ont soumis les moustiques (femelles adultes) à 42\textdegree{}C, pendant 1\,H.
		L'expression différentielle des ARNm extraits à partir des individus contrôle et soumis au stress de température ont ensuite été comparées au moyen d'une approche d'hybridation soustractive et suppressive, afin de mettre en évidence les régulations transcriptionnelles induites par le stress.
		Les résultats ont ainsi montré que le stress thermique chaud induisait chez le moustique
		% Enfin, une autre approche que l'on pourrait citer afin de mettre en évidence des différences d'expression de gènes lors d'un stress thermique consisterait en une analyse de l'ARN transcrit en des quantités différentes du contrôle, par la technique de <<~Suppression subtractive hybridization~>>\footnote{CVM : Je ne sais pas comment traduire le nom de cette technique, si il existe une traduction}.
		% Cette technique est une méthode de PCR qui amplifie spécifiquement les ADN complémentaires (obtenus à partir de l'ARN extrait des insectes soumis ou non au stress thermique) qui diffèrent de ceux di contrôle.
		% Cela permet donc de détecter les régulations transcriptionnelles induites par le stress.
		% L'étude menée ainsi sur le moustique \esp{Aedes aegypti}, un vecteur notamment du virus de la Dengue et du Chikungunya, révèle ainsi 
		une augmentation de l'expression de certaines HSPs, et d'autres protéines impliquées notament dans la synthèse protéique, et le métabolisme énergétique).

		% Autre exemple : Salvucci et al 2000

		% HSPs et stress froid
		%\cite{zhang2011}
		% Ausi Colinet 2010 qui parle des HSP et chill-coma cez droso

		% Salvucci 2000
		Enfin, un dernier exemple de reponse de <<~Heat Shock~>> concerne la mouche \esp{Bemisia argentifolii}.
		Cette étude a premis de montrer que cette mouche répondait au stress thermique (ici une exposition à 40\textdegree{}C) par une expression accrue de hsp70 et hsp90, mais aussi par une accumulation de sorbitol.
		Cette molécule est connue pour être produite en cas de stress thermique chez certains insectes, comme le puceron \cite{hendrix1998}, et a pour propriété de protéger la structure des protéines contre la dénaturation.
		Elle agirait donc comme auxiliaire des HSPs en situation potentiellement dénaturante.

		\subsubsection{Autres types de réponses physiologiques au stress thermique} % À reformuler.

		\paragraph*{}
		Les réponses physiologiques à un stress abiotique aussi important que la température ne sont pas réduits à la seule réponse de la stimulation de la production de HSPs chez l'insecte.
		En effet, il existe d'autres types de réponses physiologiques au stress thermique documentés dans la littérature, comme, pour les stress froids \cite{clark2008}, la diapause, les protéines anti-gel, le désèchement, ou pour les stress chauds la dessication, le disfonctionnement des voies respiratoires et des voies endocrines \cite{neven2000}.

		\paragraph*{}

		% Andersen 2006 (Dopamine levels in the mosquito Aedes aegypti during adult development, following blood feeding and in response to heat stress) → Résultats pas significatifs
		% Hirashima200
		Une manifestation courante chez les insectes, en réponse à un stress quel qu'il soit est la synthèse d'amines biogéniques connus comme étant des neurotransmetteurs et/ou hormones (dopamine, octopamine, \ldots) impliqués dans différents processus physiologiques de l'insecte comme par exemple la locomotion ou la métamorphose, en régulant la production d'autres hormones comme l'hormone juvénile \cite{hirashima2000}.
%		Leur action stimulatrice entraîne (complète stp en décrivant l'effet positif pour l'insecte, maintien, équilibre, … ?). 
		%Les amines biogéniques, comme la dopamine ou l'octopamine sont connus comme étant des neurotransmetteur et/ou hormones circulant dans l'hémolymphe, impliqués dans de multiples mécanismes physiologiques de l'insecte, comme la synthèse d'hormones, le processus de mue, ou la locomotion.
		% Elle est de plus connue pour être aussi une réponse non-spécifique aux stress, de part ses actions stimulantes.
		% Par exemple, dans le cas de la locomotion, sa stimulation permettrait des changements éthologiques comme une fuite de la source du stress, 
		%L'étude que nous donnerons en exemple prend comme modèle la drosophile \esp{Drosophila virilis} \cite{hirashima2000}.
		Une exposition de mouches \esp{Drosophila virilis} à un stress thermique chaud (38\textdegree{}C, durée variable) a révélé une augmentation significative des amines biogéniques, proncipalement la dopamine et l'octopamine, caractéristique de la réponse nerveuse et hormonale de l'insecte en situation de stress.
		% On fera subir à ces mouches un stress thermique (exposition à 38\textdegree{}C, pendant des durées variables), et le taux des différentes amines biogéniques sera mesuré par une technique d'HPLC (\eng{High-Pressure Liquid Chromatography}), a partir d'extraits de tissus de la tête de la drosophile.
		% Les résultats montrent une augmentation de ces amines biogéniques, signes d'une réponse nerveuse et humorale de l'insecte au stress de température.

		% Armstrong 2012 : K⁺
		Une autre réponse, non spécifique, est l'homéostasie du potassium dans l'hémolymphe \cite{armstrong2012}.
		En effet, il est admis que la chute de la concentration d'ions K\up{+}, induite notamment par une baisse de température, provoque un <<~coma~>> temporaire chez l'insecte.
		Chez la mouche \esp{Drosophila melanogaster}, il a été montré qu'un pré-traitement avec des chocs froids successifspermettait de prémunir l'insecte contre ce phénomène.
		Les mécanismes impliqués dans cette acclimatation rapide aux chocs froids sont mal connus.
		Une des pistes actuelles serait que l'effet du froid sur les pompes à potassium soient à l'origine de ce dérèglement.

		% Métabolomique : Malmendal2006
		Enfin, une étude sur \esp{Drosophila melanogaster} basée sur des techniques de métabolomiques montrent aussi des profils métaboliques significativement différents entre les mouches ayant subi un stress thermique et les témoins non stressées \cite{malmendal2006}.

% THV (Pucerons...)
