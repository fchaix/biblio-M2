%%%%%%%% 1er paragraph %%%%%%%%%%%%%%%%%%%%%%%%
% 1ère partie : expose l'aspect général du sujet (CONNU)
%%%%%%%%%%%%%%%%%%%%%%%%%%%%%%%%%%%%%%%%

% Notes : À placer : poikilotherms (inverse de l'homéotherme)

\paragraph{} % (fold)
\label{par:intro1}

Tous les organismes, dans la nature, sont confrontés à des conditions défavorables dont les conséquences peuvent être très variables dans leur intensité allant d'une perturbation de la croissance jusqu'à la mort.
Parmi les stress abiotiques, la température est un facteur majeur susceptible d'effecter le comportement des organismes.
Plus communément appelé stress thermique, il correspond à la réponse cellulaire à l'augmentation de température.
Des microorganismes aux cellules humaines, en passant par les cellules animales ou végétales, le stress thermique est universel et tous les organismes ont développé diverses stratégies pour y faire face.
Parmi les réponses caractéristiques d'une adaptation cellulaire au stress thermique, l'une des plus étudiées est la synthèse de nombreuses petites protéines appelées Hsps (Heat- Shock Proteins ou protéines de choc thermique.
Leur fonction primaire est de restaurer un environnement correct de protéines repliées (homéostasie) en assurant ainsi la protection, le maintien et la régulation de la fonction de ces protéines, indispensables à la  survie de la cellule.

% Tous les organismes dans la nature sont sans cesse confrontés à des stress abiotiques, et doivent résister à ceux-ci.
% Le stress thermique en particulier a été tres étudié, car représentant un stress à la fois concernant tous les organismes et facile à mettre en oeuvre de manière expérimentale, ce qui en a fait un type de stress que l'on pourrait califier de <<~modèle~>> dans le cadre des études du stress en général.
% Les insectes en particulier, par leur qualité de poïkilothermes\footnote{Un organisme poïkilotherme, par opposition avec les homéothermes, est un organisme qui voit naturellement sa température interne varier très fortement en fonction de celle de son environnement.},
% sont particuliérement sensibles aux changements brutaux de température.
% On retrouvera donc au sein de ce groupe des capacités d'adaptation aux stress thermiques qui, de surcroît, ont été beaucoup étudiées, en raison des fortes implications des insectes dans des problématiques d'importance économiques (ravageurs, espèces invasives…) ou touchant la santé publique (insectes vecteurs…).

%
	% Il est de plus en plus admis dans la communauté scientifique que le microbiote
	% associé à un hôte, que l'on parle de flore commensale ou symbiotique
	% \footnote{Le terme de symbiote sera utilisé ici au sens lagre, a savoir
	% comprennant à la fois les symbiotes au sens strict (bénéfice réciproque), mais
	% aussi les parasites, le point dommun étant le carractère obligatoire de la
	% relation qu'il entretient avec son hôte.}, a un impact important sur les
	% traits d'histoire de vie de son hôte \cite{feldhaar2011}. De ce constat
	% découle la notion d'holobionte \cite{rosenberg2007}, qui consiste à considérer
	% en tant qu'unité séléctve non plus le génotype d'une espèce, mais ceux,
	% combinés, de toute sa microflore et de lui-même, formant ainsi une sorte de
	% méta-génome, et potentiellement de méta-organisme sur lequel la pression de
	% sélection s'éxercerait.

	% Pour prendre l'exemple des insectes, nous pouvons par exemple citer le cas de
	% la bactérie \esp{Buchnera sp.},  endosymbiotique du puceron, jouant un rôle
	% essentiel dans le métabolisme de son hôte,  en synthétisant des acides aminés
	% que le puceron n'arriverait pas à obtenir en quantité suffisante dans la sève
	% élaborée, qui constitue sa seule source de nourriture \cite{douglas1998}.
	% \todo[inline]{Rédiger exemple du puceron qui change de couleur avec un changement de symbiote (impact sur la rédation), c’est un exemple rigolo}


%%%%%%%% 2ème paragraph %%%%%%%%%%%%%%%%%%%%%%%
% 2ème partie : précise l'aspect particulier du problème (VA VERS L'INCONNU)
%%%%%%%%%%%%%%%%%%%%%%%%%%%%%%%%%%%%%%%%

\paragraph{} % (fold)
\label{par:intro2}

Au sein du règne animal, les insectes, en leur qualité de poïkilothermes%
\footnote{Un organisme poïkilotherme, par opposition avec les homéothermes, est un organisme qui voit naturellement sa température interne varier très fortement en fonction de celle de son environnement.},
sont particulièrement sensibles aux changements brutaux de température.
De ce fait, ils représentent des organismes modèles devant répondre au défi adaptatif posé par le changement climatique.
De plus, leur forte implication dans des problématiques écologiques, économiques et sanitaires (insectes ravageurs, espèces invasives, insectes vecteurs) ont motivé les efforts de recherche pour comprendre les processus d'adaptation mis en jeu face à ces changements globaux.
Outre les réponses adaptatives de l'insecte per se, des travaux récents ont montré l'existence d'interférence symbiotique dans les processus adaptatifs rapides des insectes \cite{feldhaar2011}.
En effet, parmi toutes les associations du vivant, les liens fonctionnels qui unissent les arthropodes et leur flore microbienne ou microbiote2 sont le fruit d'une évolution concertée.
Il est maintenant admis que ces systèmes symbiotiques doivent être considérés dans leur ensemble pour appréhender les phénotypes étendus qui résultent de ces interactions multipartenaires.
Jusqu'à présent, la plupart des travaux a concerné l'étude des  associations  symbiotiques  des  insectes  ravageurs  de  cultures.
Ces  modèles  ont  révélé  l'implication  de  la  flore  bactérienne  des  hôtes  dans  des  fonctions  clés  comme  la nutrition, la reproduction et la protection contre les ennemis naturels \cite{dillondillon}.
Par exemple, il a été montré que les bactéries symbiotiques influeraient sur la préférence d'accouplement des drosophiles en modifiant la composition des phéromones produites par les insectes \cite{sharon2010}.
Un autre exemple révélateur est la mise en évidence du rôle des bactéries symbiotiques dans la coloration des pucerons, un trait important dans les interactions avec les prédateurs \cite{tsuchida2010} ou encore leur métabolisme via la synthèse d'acides aminés essentiels que le puceron est incapable d'obtenir en quantité suffisante dans la sève élaborée, qui constitue sa seule source de nourriture \cite{douglas1998}.
Tous ces exemples de phénotypes étendus alimentent la théorie selon laquelle l'hôte est indissociable des microorganismes qu'il héberge.
De ce constat découle la notion d'holobionte \cite{rosenberg2007}, qui consiste à considérer en tant qu'unité sélective non plus le génotype d'une espèce, mais ceux, combinés, de toute sa microflore et de lui-même, formant ainsi une sorte de méta-génome, et potentiellement de méta-organisme sur lequel la pression de sélection s'exercerait.

% Parmi les approches permettant d'essayer de comprendre les mécanismes de réponse aux stress, il en est une de plus en plus en vogue : celle consistant à prendre en compte la communauté microbienne associée à l'insecte, comme partie prennante de la réponse à l'environement.
% En effet, Il est de plus en plus admis dans la communauté scientifique que le microbiote associé à un hôte, que l'on parle de flore commensale ou symbiotique%
% \footnote{Le terme de symbiote sera utilisé ici au sens lagre, a savoir comprennant à la fois les symbiotes au sens strict (bénéfice réciproque), mais aussi les parasites, le point dommun étant le carractère obligatoire de la relation qu'il entretient avec son hôte.},
% a un impact important sur les traits d'histoire de vie de son hôte \cite{feldhaar2011}.
% De ce constat découle la notion d'holobionte \cite{rosenberg2007}, qui consiste à considérer en tant qu'unité séléctve non plus le génotype d'une espèce, mais ceux, combinés, de toute sa microflore et de lui-même, formant ainsi une sorte de méta-génome, et potentiellement de méta-organisme sur lequel la pression de sélection s'éxercerait.

% Pour prendre l'exemple des insectes, nous pouvons par exemple citer le cas de la bactérie \esp{Buchnera sp.},  endosymbiotique du puceron, jouant un rôle essentiel dans le métabolisme de son hôte,  en synthétisant des acides aminés que le puceron n'arriverait pas à obtenir en quantité suffisante dans la sève élaborée, qui constitue sa seule source de nourriture \cite{douglas1998}.


%
	% Parmi les insectes, justement, se retrouve bon nombre d'espèces considérées
	% comme invasives, voir extrêmement invasives comme le moustique tigre
	% \esp{Aedes albopictus}. Leur carractère invasif est extrêmement lié à une
	% capacité à s'acclimater à une variation climatique brutale, car la température
	% environementale est l'un des facteurs abiotiques environementaux affectant le
	% plus les insectes.

	% % En plus d'être invasifs, ils sont méchants. 

	% L'intérêt de l'étude du déterminisme de cette capacité d'adaptation est
	% d'autant plus important que ces insectes invasifs sont parfois vecteurs
	% maladies, comme c'est le cas d'\esp{Aedes albopictus}, cité précédement, ce
	% qui fait de l'étude de ces insectes une problématique majeure de santé
	% publique \cite{schaffner2013}.

%%%%%%%% 3ème paragraph %%%%%%%%%%%%%%%%%%%%%%%%
% 3ème partie : indique le but du travail et un aperçu des résultats 
% (POSE LA QUESTION)
%%%%%%%%%%%%%%%%%%%%%%%%%%%%%%%%%%%%%%%%%%

\paragraph{} % (fold)
\label{par:intro3}

Dans un contexte où le changement climatique est aujourd'hui une réalité, l'étude de l'adaptation en réponse au stress thermique est une question importante en biologie évolutive.
L'objectif de ce mémoire est de faire un état des lieux des connaissances sur les processus adaptatifs des insectes face au stress thermique.
Dans une première partie seront décrits les mécanismes généraux mis en place par les insectes en réponse au stress thermique.
Des exemples récents ayant montré l'implication du microbiote dans l'adaptation des insectes à leur environnement, la deuxième partie de ce mémoire décrira les principaux mécanismes connus par lesquels le microbiote peut influer sur cette réponse.

% Il serait donc intéressant d'envisager l'étude de cette résistance au stress thermique sous un nouvel angle, celui qui considère comme unité évolutive non plus l'insecte seul, mais l'ensemble que forme l'hôte insecte et ses bactéries symbiotiques et commensales.
% Autrement dit, nous chercherons à déterminer l'impact qu'a la microflore des insectes sur leurs capacités de résistance au stress thermique.

%
	% Il serait donc intéressant d'envisager l'étude de cette résistance au stress
	% thermique sous un nouvel angle, celui qui considère comme unité évolutive non
	% plus l'insecte seul, mais l'ensemble que forme l'hôte insecte et ses bactéries
	% symbiotiques et commensales. Autrement dit, nous chercherons à déterminer
	% l'impact qu'a la µflore des insectes sur leurs capacités de résistance au
	% stress thermique.

% Nous décrirons dans un premier temps les mécanismes de réponse au stress
% thermique propres à l'insecte, puis dans un second temps, 
% nous décrirons les principaux mécanismes par lesquels le microbiote peut influer sur cette réponse, avec des exemples pour chaque mécanisme.

