\chapter{Interférences du microbiote bactérien de l'insecte dans la réponse aux stress de température} % (fold)

\paragraph*{}
Nous avons vu différentes voies de réponse de l'insecte aux stress thermiques, en réduisant le système à l'insecte seul.
Or nous savons que dans la nature, la sélection s'exerce sur l'ensemble formé par l'hôte et son microbiote, il s'agit de la notion d'holobionte, défini précédemment.
Dans un système en constante intéraction entre les partenaires, il est intuitif de supposer que le microbiote puisse jouer un rôle, plus ou moins directement, dans la réponse au stress thermique, de laquelle dépend la fitness de l'holobionte ?

La diversité des mécanismes pouvant mettre en relation le microbiote bactérien et leur hôte dans un phénomène de réaction à un stress thermique est très grande, et fait surtout l'objet de recherches toujours actives.
Nous décrirons dans cette partie les mécanismes les plus fréquemment retrouvées à l'heure actuelle.

\section{Stimulation de l'expression de gènes de l'hôte}

Dans le premier type de relation que nous allons citer, la présence de la bactérie au contact des tissus de l'insecte stimule l'expression de protéines de stress, ou du moins de thermotolérance, quelque soit le contexte environnemental.
En effet, l'infection bactérienne est connue comme étant un déclencheur d'une réponse de heat shock (voir tableau 1) \cite{deitch1995}.
L'hôte est donc dans un état de stress <<~artificiel~>>, y compris dans un environnement non stressant, ce qui lui permet d'être <<~préparé~>> à un éventuel choc thermique, et donc de mieux y résister.

Notre premier exemple concernera le couple symbiotique que forme la mouche \esp{Bemisia tabaci}, un ravageur de culture et vecteur de virus phytopathogènes, et son symbiote secondaire \esp{Rickettsia sp.} \cite{brumin2011}.
Cette étude a mis en évidence un mécanisme conduisant à une meilleure résistance de la mouche au stress thermique, lorsqu'elle est associée à \esp{Rickettsia sp.}.
Partant du constat que les populations associées à \esp{Rickettsia sp.} étaient plus résistants aux chocs thermiques chauds (incubation des femelles adultes à différentes températures, pendant 3h30), les chercheurs ont mesuré l'expression de gènes connus comme faisant partie de la réponse thermotolérante.
Les résultats montrent que les HSPs, contrairement à ce à quoi l'on pourrait s'attendre, ne sont pas significativement induits par la présence de \esp{Rickettsia sp.}, mais qu'en revanche diverses protéines associées au cytosquelette sont nettement induites.
Or, on sait que le cytosquelette, de part sa structure protéique, est l'un des organes cellulaire le plus rapidement touché par les chocs thermiques.
\esp{Rickettsia sp.} semblerait capable d'induire la sur-expression de protéines du cytosquelette de leur hôte, induisant ainsi une meilleure résistance aux stress thermiques.

% Tang 2012 → Je suis un con, j'avais mal compris la publi, c'était pas ça.

\section{La réponse au stress du microbiote affecte la réponse de l'hôte}

Le microbiote est, tout comme son hôte, affecté par les variations de température, et va répondre à ces chocs afin d'y résister.
Les molécules produites par cette réponse peut alors, du fait de la grande proximité hôte-microbiote, profiter à l'hôte.

Notre premier exemple portera sur l'association entre \esp{Bushnera sp.} et le puceron \esp{Acyrthosiphon pisum}.
Une étude a montré qu'une mutation, existante dans la nature, d'une seule nucléotide dans le promoteur d'une petite HSP de \esp{Bushnera} induisait chez l'hôte du mutant une résistance moindre aux choc thermiques \cite{dunbar2007}.
Cette mutation dans le promoteur de \gene{ibpA}, un gène codant pour une petite HSP, change le contexte dans lequel l'expression de ce gène va se produire. Dans la souche sauvage, \gene{ibpA} s'exprimme en cas de choc thermique chaud, et chez les mutants, \gene{ibpA} va s'exprimer dans un contexte de température froide.
Les observations de cette étude montrent que la population de pucerons infectés par la souche sauvage avait un meilleur taux de survie dans un contexte de choc thermique chaud, mais que l'inverse était vrai aussi : les pucerons qui portent les bactéries possédant l'allèle muté sont plus aptes à survivre dans un environement froid.

% Exemple GroEL
Notre deuxième exemple porte sur les familles de protéines hsp60/hsp10, très bien conservées dans le vivant car étant présentes dans tous les groupes du vivant, aussi bien dans le génome eucaryote et archéen (le complexe Tcp-1), que dans celui des bactéries (GroEL/GroES) et les organelles des eucaryotes.
Cette similitude étant telle que l'hypothèse d'une origine commune est plutôt bien appuyée, et constitue même une des bases de la théorie endosymbiotique des organelles des eucaryotes \cite{gupta1995}.
Dans ce deuxième exemple, il est montré que la protéine GroEL est sur-exprimée par l'endosymbionte \esp{Blochmannia floridanus} en cas de stress, et qu'elle diffuse dans la cellule de son hôte, la fourmi \esp{Campontus floridanus} \cite{stoll2009}.
L'endosymbionte jouerait ici un rôle similaire à celui des mitochondries de l'hôte.
% Une fois dans l'hémolymphe, de la même façon que le feraient les hsp60 produites par les organelles, la chaperonne exercerait ses fonctions de renaturation sur les protéines de l'hôte.

\section{Rôle de la dynamique de densité des populations symbiotiques}

Les conditions climatiques sont souvent corrélées à un changement parfois radical dans la structure de la population bactérienne associée à un hôte, ce que l'on peut assimiler à une <<~mutation~>> de l'holobionte.
Il a par exemple été montré que les populations de pucerons vivant dans des milieux soumis à des fluctuations régulières de température présentaient une prévalence de leurs symbiontes secondaires bien plus élevés que ceux de pays plus tempérés \cite{harmon2009}.

De même, Feder \textit{et al.} suggère dans une revue un lien direct entre le stress thermique régulant la communauté de \esp{Wolbachia} dans la drosophile et la capacité de ces endosymbiotes d'induire l'incompatibilité cytoplasmique, modifiant ainsi la fitness de l'holobionte \cite{feder1999}. 

Une autre étude \cite{bordenstein2011} concernant la relation tripartite entre une guêpe (\esp{Nasonia vitripennis}), l'endosymbionte \esp{Wolbachia} et le bactériophage WO montre que le stress thermique a non seulement un effet sur la densité de \esp{Wolbachia}, l'induction de la phase lytique de WO, mais aussi sur le seuil de densité a partir duquel \esp{Wolbachia} induit l'incompatibilité cytoplasmique sur son hôte.
%Des populations de la guêpe ont été élevées sous des températures différentes (18°C, 25°C et 32°C). La densité de Wolbachia a été mesurée par RT-qPCR
% Plus taaaaard !!!
%%%% Les publis que j'ai pas encore mis
%\cite{montllor2002}
