\usepackage[francais]{babel}
\usepackage{fontspec}
    \setmainfont[%
	Numbers=OldStyle,
	Ligatures={
		TeX,
		Common,
		Rare,
		Historical
		}
	]{Linux Libertine O}
    % \setmonofont{Ubuntu Mono}
    \setsansfont{Linux Biolinum O}

\KOMAoptions{DIV=last} 
% La définition de DIV, dans le cas de calc, doit se déclarer après la
% déclaration de la police.

\usepackage[%
    hidelinks=true
    colorlinks=false
    pdfauthor={François Chaix},
    pdftitle={Document de travail — recherches biblio François Chaix},
    pdfdisplaydoctitle=true, % Display document title instead of filename in title bar
    pdfsubject={plop},
    pdfkeywords={plop},
    pdfproducer={LuaTeX, avec le package hyperref},
    pdfcreator={LuaTex},
    linktocpage=false,
    pdfinfo={pouet ?},
    pdflang={fr-FR},
    unicode=true,
    verbose=true
    ]{hyperref}

\usepackage{lipsum}

% Cet environnement Figure ne servira que si on est obligé d’insérer une image
% dans un environnement multicol. Mais c’est moche.
\newenvironment{Figure}
  {\par\medskip\noindent\minipage{\linewidth}}
  {\endminipage\par\medskip}

%%%%%%%% Concernant la biblio %%%%%%%%%%%
\usepackage[
%	style=reading,
	natbib=true,
	backend=biber
	]{biblatex}
\addbibresource{bib.bib}

%%%%%%%% Concernant le document %%%%%%%%%
\title{Rôle de la communauté symbiotique bactérienne dans la réponse des insectes au stress thermique}
\author{François \textsc{Chaix}}
