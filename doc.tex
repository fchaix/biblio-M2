\documentclass[%
	a4paper,
	10pt,
	linktocpage=true,
	oneside,
	twocolumn,
	DIV=calc, % DIV règle les proportions corps de texte-marges. Classic, c’est pour suivre les canons du Moyen-âge. Pardonnez-moi, prince.
%	twocolumn % Inutile avec le package multicol.
	]{scrartcl}
\usepackage[francais]{babel}
\usepackage{fontspec}
    \setmainfont[%
	Mapping=tex-text,
	Numbers=OldStyle,
	Ligatures={
		TeX,
		Common,
		Rare,
		Historical
		}
	]{Linux Libertine O}
    \setmonofont{Ubuntu Mono}
    \setsansfont{Linux Biolinum O}
\KOMAoptions{DIV=last}
\usepackage[%
    hidelinks=true
    colorlinks=false
    pdfauthor={François Chaix},
    pdftitle={Document de travail — recherches biblio François Chaix},
    pdfdisplaydoctitle=true, % Display document title instead of filename in title bar
    pdfsubject={plop},
    pdfkeywords={plop},
    pdfproducer={LuaTeX, avec le package hyperref},
    pdfcreator={LuaTex},
    linktocpage=false,
    pdfinfo={pouet ?},
    pdflang={fr-FR},
    unicode=true,
    verbose=true
    ]{hyperref}
\usepackage{lipsum}

% Cet environnement Figure ne servira que si on est obligé d’insérer une image
% dans un environnement multicol. Mais c’est moche.
\newenvironment{Figure}
  {\par\medskip\noindent\minipage{\linewidth}}
  {\endminipage\par\medskip}

%%%%%%%% Concernant la biblio %%%%%%%%%%%
\usepackage[
%	style=aem,
	natbib=true,
	backend=biber
	]{biblatex}
\addbibresource{bib.bib}

%%%%%%%% Concernant le document %%%%%%%%%
\title{Mon super titre que je trouverais plus tard}
\author{François \textsc{Chaix}}

%%%%%%%% Début du document %%%%%%%%%%%%%%
\begin{document}
\nocite{*}

\twocolumn[
\begin{@twocolumnfalse}
	\maketitle
	\begin{abstract}

Ceci est un document de travail pour la rédaction de mon mémoire
bibliographique de recherche pour mon M2. La version twocolumns n’est
évidement pas la version finale (les gens fixant les règles de présentation
ayant des goûts douteux en terme de typographie), mais la version dédiée au
travail sur le fond (ça fait moins de pages à imprimer, et y’a pas à dire,
c’est plus lisible.). Merci de votre compréhension.

\end{abstract}
	\tableofcontents
  \vspace{1cm}
\end{@twocolumnfalse}
]

	% \chapter{chapitre}
\section{section}
	\lipsum[1]
\subsection{subsection}
	\lipsum[1]

	\begin{figure}[tb]
	\begin{center}
		\includegraphics[height=7cm]{troll.png}
	\end{center}
	\caption{Problem ?}
	\label{fig:troll}
\end{figure}
	
\printbibliography

\end{document}
