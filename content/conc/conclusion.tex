
\section*{\textsc{Conclusion}}
\addcontentsline{toc}{section}{\textsc{Conclusion}}

La réponse aux stress thermiques chez les insectes constitue un processus complexe par la multiplicité des mécanismes mis en jeu.
Outre les mécanismes physiologiques et moléculaires, des stratégies comportementales (fuite, migrations, construction d'abris\ldots) sont également mises en place par les insectes pour s'adapter aux conditions environnementales défavorables, témoignant ainsi de leur plasticité hors-norme. 
Il est d'ailleurs admis que de nombreuses espèces d'insectes ont vu leur aire de répartition géographique s'étendre en répercussion aux changements climatiques.
Les moustiques vecteurs illustrent parfaitement ce phénomène puisque les observations récentes décrivent une migration de certaines espèces vers les pays aux climats plus froids que leur région d'origine.
La prise en compte récente du concept d'holobionte permet aujourd'hui d'appréhender sous un nouvel angle, plus intégratif, les processus adaptatifs des insectes aux changements de conditions climatiques.
Comme documenté dans cette synthèse bibliographique, il a notamment été montré que les symbiotes bactériens peuvent contribuer dans la thermotholérance de certains insectes phytophages hôtes.
Bien que les moustiques hématophages hébergent également un microbiote bactérien riche et varié, son rôle dans la thermotolérance de ces insectes vecteurs de nombreux pathogènes à l'homme et aux animaux demeure à explorer.

\paragraph*{}
Mon stage de master s'inscrit justement dans ce contexte, et il consistera à mesurer les interférences entre la réponse du moustique tigre \esp{Aedes albopictus} au stress froid et l'un de ses symbiotes secondaires cultivables, la bactérie \esp{Acinetobacter calcoaecitus}. 

% La réponse aux stress thermiques constitue un processus complexe, de part la multiplicité des mécanismes mis en jeu. 
% Certaines de ces réponses, comme par exemple, les réponses éthologiques (fuite, migrations, construction d'abris\ldots)  n'ont pas été évoquées dans ce mémoire et constituent pourtant une réaction non négligeable de l'insecte face à un stress de ce type.
% De plus, nous n'avons évoqué dans ce mémoire uniquement la réponse de l'insecte au stress chaud, en négligeant volontairement les aspects liés au stress froid, pourtant primordial dans la biologie des insectes soumis aux fluctuations environnementales.
% % Nous n'avons pas, par exemple, évoqué les réponses éthologiques (fuite, migrations, construction d'abris...), et sans doute passé sous silence certains éléments physiologiques.
% Jusqu'à présent, l'étude de la réponse au stress thermique a souvent été considérée à l'échelle même de l'individu ou des populations.
% Avec la prise en compte récente du concept d'holobionte, l'organisme n'est plus considéré comme un individu isolé mais comme une communauté d'espèces (lui-même et tous les organismes qu'il renferme).
% % Cela reflète la grande diversité des réponses à ces stress, et le fait que de part son importance dans les écosystèmes vivants, ces différentes réponses ont déjà été étudiées sous beaucoup d'angles.
% Cette nouvelle dimension permet aujourd'hui d'appréhender sous un nouvel angle, plus intégratif, les processus adaptatifs des insectes aux changements de conditions climatiques.
% Or ces derniers jouent un rôle important dans les processus de migration observés actuellement chez les moustiques vecteurs, notamment vers les pays aux climats plus froids que leurs régions d'origine.
% Un exemple-type serait celui du moustique tigre, \esp{Aedes albopictus}, originaire d'Asie du Sud-Est, colonisant peu à peu nos régions tempérées, et dont la tolérance au froid pourrait être un des points clés pour expliquer son adaptation.
% De façon intéressante, il a été montré que ces moustiques hébergent un microbiote bactérien riche et varié dont le rôle dans la thermotolérance reste encore à déterminer.
% % Il n'a cependant pas été beaucoup étudié sous l'angle de l'holobionte, notion qui permet pourtant d'ajouter un grand facteur de plasticité dans notre compréhesion de l'interface individu-environement.
% % Cependant la prise de conscience de cette dimention est aujourd'hui en plein essor, notamment en raison de l'importance grandissante des épidémies d'arboviroses.

% C'est justement dans ce contexte que mon stage de master se déroulera, il consistera à mesurer les interférences entre la réponse du moustique tigre \esp{Aedes albopictus} au stress froid et l'un de ses symbiotes secondaires cultivables, la bactérie \esp{Acinetobacter calcoaecitus}.

