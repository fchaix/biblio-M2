\chapter{Réponses de l'insecte aux stress de température}

% Préambule accilmatation vs. adaptation
En biologie tout stress appelle une réponse de l'organisme pour maintenir l'équilibre biologique dans un état fonctionnel.
Deux types de réponses, souvent confondues à tort, sont observées lors d'un stress : l'adaptation et l'acclimatation.
La différence se situe essentiellement sur l'échelle de temps durant laquelle se déroule le phénomène.
Dans le cas de l'adaptation, les organismes seront confrontés à un changement environnemental sur l'échelle de plusieurs générations.
Ce changement sur le long terme va introduire une pression de séléction qui va à plus ou moins long terme modifier le génome de la lignée (ou la composition de l'holobionte), qui sera plus adaptée à ce nouvel environnement.
Das le cas de l'acclimatation, l'échelle de temps est réduite à celle de la vie d'un individu. La réponse au changement environnemental sera dépendant des capacités intrinsèques de l'individu ou de l'holobionte en question, et ces changements seront réversibles.
On parle aussi de plasticité adaptative de l'individu pour désigner cette capacité d'acclimatation.
Ce mémoire sera plus particulièrement consacré à l'étude des
phénomènes d'acclimatation suite à un stress ponctuel dans le temps, exercé à l'échelle des individus.

	\section{Les protéines de choc thermique, clé de voûte de la réponse acclimative}

	La famille de protéines de réponse au stress que nous allons évoquer dans cette partie ne concerne pas uniquement les insectes.
	Les HSP (\eng{Heat Shock Proteins}) constituent en effet la réponse la plus universelle au stress thermique qu'il est possible de trouver, des bactéries aux vertébrés supérieurs.
	Elles sont par conséquent de loin les plus documentées dans la littérature.
	Leur découverte remonte à 1974 où Tissière \textit{et al.} les mit en évidence par éléctrophorèse dans des extraits tissulaires de drosophiles ayant subi un stress chaud \cite{tissieres1974}.

	\section{Exemples de réponse chez les insectes}

	% HSPs et stress froid
	\cite{zhang2011}

		\subsection{Induction de lé réponse de <<~Heat-Shock~>>}

		\subsection{Autres types de réponses} % À reformuler.

