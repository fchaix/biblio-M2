%%%%%%%% 1er paragraph %%%%%%%%%%%%%%%%%%%%%%%%
% 1ère partie : expose l'aspect général du sujet (CONNU)
%%%%%%%%%%%%%%%%%%%%%%%%%%%%%%%%%%%%%%%%

La résistance au stress, en particulier thermique, est une carractéristique
principale des espèces dites invasives, notamment chez les insectes. La
compréhension des mécanismes sous-jascents de cette résistance présente donc
un enjeu majeur, tant pour des raisons scientifiques que de santé publique. En
effet, certaines espèces invasives sont extrêmement problématiques, car
pouvant être des vecteurs de maladies souvent virales, comme par exemple
\esp{Aedes albopictus}, vecteur du Chikungunya, de la Dengue et d'un grand
nombre d'autres virus.

%%%%%%%% 2ème paragraph %%%%%%%%%%%%%%%%%%%%%%%
% 2ème partie : précise l'aspect particulier du problème (VA VERS L'INCONNU)
%%%%%%%%%%%%%%%%%%%%%%%%%%%%%%%%%%%%%%%%

Or, il a été montré\footnote{ref. nécessaire} que les bactéries symbiotiques
des insectes influent e façon significative sur la biologie de leurs hôtes,
modifiant leurs capacités d'adaptation, de résistance à divers stress, et même
dans certains cas leurs carractéristiques reproductives.

%%%%%%%% 3ème paragraph %%%%%%%%%%%%%%%%%%%%%%%%
% 3ème partie : indique le but du travail et un apercu des résultats 
% (POSE LA QUESTION)
%%%%%%%%%%%%%%%%%%%%%%%%%%%%%%%%%%%%%%%%%%

Nous nous intéresserons donc dans ce mémoire au rôle que peut avoir la
communauté symbiotique des insectes dans leur résistance au stress thermique.
