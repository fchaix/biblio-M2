\cite{federhoffmann1999, zhang2011}

Le terme HSP (\eng{Heat Shock Proteins}) désigne une grande famille de
protéines initialement décrites comme étant induite spécifiquement par un
stress thermique chaud. Par la suite, elles se sont avérées jouer un rôle bien
plus large dans la réponse générique aux stress de diverses
natures\cite{sorensen2003}.

% Rôles

Ces protéines font partie du grand groupe des protéines chaperonnes. Les
protéines chaperonnes jouent des rôles multiples tournent tous autour de la
gestion de la conformation des protéines, de leur repliement, leur appariement
dans des complexes protéiques, ou encore le transport.  Leur rôle est crucial
dans le processus de formation de protéines, y compris dans un environement
non stressant. Cependant, lors d'un stress, le besoin de réparation rapide des
structures endomagées par ce stress se fait plus fort, d'où l'existance de
cette sous-famille de chaperonnes que sont les HSPs.

% Fort taux de concervation

Les HSPs ont un taux de concervation dans le vivant extrêmement important, ce
qui fait que leurs gènes sont souvent califiées de gènes dits <<~de ménage~>>
(\eng{Housekeeping genes} en anglais)

% Familles

Ces HSPs sont classiquement classées dans des grandes familles en fonction de
leur poids moléculaire (par exemple, les protéines de la famille de Hsp70 ont
un poids moléculaire d'environ 70\,kDa), mais de récentes tentatives de
clarification de cette nomenlature tendraient vers des noms ne faisant pas
allusion au poids moléculaire\footnote{Kampinga et al., 2009. À voir si je le
cite ou pas, juste pour ça...}


\input{content/figures/tab_HSP}
