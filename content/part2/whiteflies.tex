\paragraph{description} % (fold)
\label{par:description_whitefly}

\esp{Rickettsia sp.} est un symbionte secondaire de esp{Bemisia tabaci}, un
diptère causant des maladies au tabac. Une étude \cite{brumin2011} a récement
mis en évidence un mécanisme conduisant à une meilleure résistance de la
mouche au stress thermique, lorsqu'elle est associée à sa bactérie
symbiotique.

\paragraph{Processus mis en jeu} % (fold)
\label{par:process_whitefly}

La présence de la bactérie dans les tissus de l'hôte induit une expression de
gènes associés au stress, de façon permanente, même dans des conditions
thermiques favorables. Cette présence continue de protéines de stress joue un
rôle préventif : Lors d'un stress thermique, l'insecte aura déjà exprimmé ses
gènes de résistance et résistera donc mieux au froid.
