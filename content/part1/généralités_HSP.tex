\cite{federhoffmann1999, zhang2011}

Le terme HSP (\eng{Heat Shock Proteins}) désigne une grande famille de
protéines initialement décrites comme étant induite spécifiquement par un
stress thermique chaud. Par la suite, elles se sont avérées jouer un rôle bien
plus large dans la réponse générique aux stress de diverses
natures\cite{sorensen2003}.

% Rôles

Ces protéines, hautement concervées dans le groupe des
eucaryotes\footnote{citer}, font partie du grand groupe des protéines
chaperonnes. Les protéines chaperonnes jouent des rôles multiples tournent
tous autour de la gestion de la conformation des protéines, de leur
repliement, leur appariement dans des complexes protéiques, ou encore le
transport.  Leur rôle est crucial dans le processus de formation de protéines,
y compris dans un environement non stressant. Cependant, lors d'un stress, le
besoin de réparation rapide des structures endomagées par ce stress se fait
plus fort, d'où l'existance de cette sous-famille de chaperonnes que sont les
HSPs.

% Familles

Ces HSPs sont classiquement classées dans des grandes familles en fonction de
leur poids moléculaire (par exemple, les protéines de la famille de Hsp70 ont
un poids moléculaire d'environ 70\,kDa), mais de récentes tentatives de
clarification de cette nomenlature tendraient vers des noms ne faisant pas
allusion au poids moléculaire\footnote{Kampinga et al., 2009. À voir si je le
cite ou pas, juste pour ça...}


\begin{table}
\begin{center}
\begin{tabular}{ l r }
	Gène & fonction \\
	\hline
	plop & pouet\\
\end{tabular}
\caption{Super tableau :)}
\label{tab:tab1}
\end{center}
\end{table}

