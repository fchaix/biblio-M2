Outre les HSP, nous pouvons citer d'autres phénomènes constituant une réponse au stress thermique chez l'insecte.

Tout d'abord, il est important de rappeler que le stress thermique, avant d'exercer ses effets délétères sur la physiologie, notemment les peotéines, induitdans un premier temps une réaction d'ordre nerveuse et hormonale, qui servira de signal afin de signaler la perturbation.
La dopamine est un eurotrnsmetteur de l'insecte associé entre autres aux signaux de type stress, et est sollicité lors d'un stress thermique \cite{andersen2006}

% Il y a aussi (Armstrong et al), avec cette histoire de K+ dans le cerveau…

\paragraph{}

Les chocs de température n'affectent pas uniquement la conformation des protéines.
Un autre point clé physiologique est aussi sévèrement touché : le système nerveux.
La perturbation termique va affecter les mouvements ioniques (K⁺) nécessaires aux potentiels d'action, va notament provoquer un phénomène de «coma» thermique (\eng{chill coma} en anglais).
