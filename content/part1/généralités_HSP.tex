\cite{federhoffmann1999, zhang2011}

Le terme HSP (\eng{Heat Shock Proteins}) désigne une grande famille de
protéines initialement décrites comme étant induite spécifiquement par un
stress thermique chaud. Par la suite, elles se sont avérées jouer un rôle bien
plus large dans la réponse générique aux stress de diverses
natures\footnote{citer}.

% Rôles

Ces protéines, hautement concervées dans le groupe des
eucaryotes\footnote{citer}, ont des rôles divers, mais la plupart d'entre
elles sont des protéines chaperonnes, dont le rôle est de replier les
protéines dénaturées et/ou naïves\footnote{vérifier traduction. Protéines tout
juste traduites, pas encore avec forme finale et fonctionnelle.}. Les autres,
moins nombreuses, jouent des rôles dans le trafic
intracellulaire\footnote{chercher autres rôles}.


% Familles

Ces HSPs sont classiquement classées dans des grandes familles en fonction de
leur poids moléculaire (par exemple, les protéines de la famille de Hsp70 ont
un poids moléculaire d'environ 70\,kDa), mais de récentes tentatives de
clarification de cette nomenlature tendraient vers des noms ne faisant pas
allusion au poids moléculaire\footnote{Kampinga et al., 2009. À voir si je le
cite ou pas, juste pour ça...}


\begin{table}
\begin{center}
\begin{tabular}{ l r }
	Gène & fonction \\
	\hline
	plop & pouet\\
\end{tabular}
\caption{Super tableau :)}
\label{tab:tab1}
\end{center}
\end{table}

