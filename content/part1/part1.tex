\chapter{Réponses de l'insecte aux stress de température}

% Préambule accilmatation vs. adaptation
En biologie tout stress appelle une réponse de l'organisme pour maintenir l'équilibre biologique dans un état fonctionnel.
Deux types de réponses, souvent confondues à tort, sont observées lors d'un stress : l'adaptation et l'acclimatation.
La différence se situe essentiellement sur l'échelle de temps durant laquelle se déroule le phénomène.

	\section{Les protéines de choc thermique, clé de voûte de la réponse acclimative}

	\section{Exemples de réponse chez les insectes}

		\subsection{Induction de lé réponse de <<~Heat-Shock~>>}

		\subsection{Autres types de réponses} % À reformuler.

