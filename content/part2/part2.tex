\chapter{Interférences du microbiote dans la réponse au stress de température} % (fold)
	\label{sec:implicationµbiote}

De manière générale, il est souvent constaté que les conditions stressantes favorisent les relations de mutualisme \cite{meadows2013}.
Nous sommes donc en droit d'émettre l'hypothèse selon laquelle les partenaires mutualistes joueraient un rôle significatif dans la réponse au stress.

La diversité des mécanismes pouvant mettre en relation le microbiote bactérien
et leur hôte dans un phénomène de réaction à un stress thermique est très
grande, et fait surtout l'objet de recherches toujours actives. Nous décrirons
dans cette partie les mécanismes les plus fréquemment retrouvées à l'heure
actuelle.


	\section{Interférences directes} % (fold)
		\label{sec:direct}

\subsection{Stimulation de gènes de l’hôte} % (fold)
\label{sub:genes_hote}

Dans ce type de relation, la présence de la bactérie dans les tissus de
l'insecte stimule l'expression de protéines de stress, quelque soit le
contexte environemental. Les protéines de stress sont donc exprimées de façon
<<artificiellemnt>> constitutive, ce qui permet à l'insecte de résister de façon
plus efficace à un stress brutal et rapide.

\subsection{Exemples} % (fold)
\label{sub:exemples}

% Exemple Rickettsia (brumin2011)

Notre premier exemple concernera le couple symbiotique que forme la mouche
\esp{Bemisia tabaci}, un ravageur de culture et vecteur de virus
phytopathogènes, et son symbiote secondaire \esp{Rickettsia sp.}. Une étude
\cite{brumin2011} a récement mis en évidence un mécanisme conduisant à une
meilleure résistance de la mouche au stress thermique, lorsqu'elle est
associée à sa bactérie symbiotique. Le processus mis en jeu dans cette
résistance est celui décrit en \ref{sub:processus_mis_en_jeu}, a savoir que la
présence de la bactérie dans les tissus de la mouche induit une expression de
gènes nécessaires à la thermotolérance (notamment ici des gènes codant pour
des éléments du cytosquelette), même dans des insectes non-soumis à un stress
thermique.
Plus spécifiquement, une autre étude \cite{tang2012} montre une induction de HSP70 par l'introduction de protéines provenant de cultures d'\esp{Escherichia coli} et \esp{Staphilococcus aureus} dans la larve de la mouche domestique (\esp{Musca domestica}).

	\section{Interférences indirectes} % (fold)
	 	\label{sec:indirect}

\subsection{La réponse au stress du microbiote affecte la réponse de l'hôte} % (fold)
\label{sub:groel}

% Exemple Bushnera et mutation gène ibpA (dunbar2007)

Notre premier exemple portera sur l'association entre \esp{Bushnera sp.} et le
puceron \esp{Acyrthosiphon pisum}. Une étude \cite{dunbar2007} a montré que la mutation d'une seule nucléotide dans le promoteur d'une petite HSP de \esp{Bushnera} induisait
chez l'hôte du mutant une résistance moindre aux choc thermiques.

Le cas de l'expression d'une protéine de choc thermique GroEL par les
organelles issues d'une ancienne endosymbiose (voir partie
\ref{ssub:la_famile_hsp60}) très similaire à celle produite par les bactéries
suggère la possibilité d'une utilisation de l'expression bactérienne de
protéines de réponse au stress, par leur hôte. On observe effectivement ce
type de mécanisme assez souvent, nous en donnerons quelques exemples.

% Exemple GroEL

Dans notre deuxième exemple, la protéine GroEL est sur-exprimée par
l'endosymbionte \esp{Blochmannia floridanus} en cas de stress, et qui diffuse
dans l'hémolymphe de son hôte, la fourmi \esp{Campontus floridanus}.
Une fois dans l'hémolymphe, les protéines de stress bactériennes peuvent
exercer leurs fonctions de chaperonne sur les protéines dénaturées de l'hôte
(\cite{stoll2009}). 
% \todo[inline]{Peut-être trouver un autre exemple, celui-ci n'est
% pas clairement le propos de la publi}





\subsection{Dynamique de densité des populations symbiotiques} % (fold)
\label{sub:taillepop}

Les contitions climatiques sont souvent corrélées à un changement parfois
radical dans la nature de la population bactérienne associée à l'hôte. Il a
par exemple été montré \cite{harmon2009} que les populations de pucerons
issues de milieux soumis à des chocs thermiques réguliers présentaient une
prévalence de leurs symbiontes secondaires bien plus élevés que ceux de pays
plus cléments.


De plus, il n'est pas rare qu'au sein d'un même hôte plusieurs populations de symbiotes
cohabitent. Il se met alors en place des intéractions classiques entre ces
populations, pouvant aller de l'antagonisme au commensalisme. Par exemple, il
a été montré \cite{montllor2002} que la présence de symbiotes secondaires
comme \esp{Rickettsia sp.} protégeait les bactériocytes\footnote{Les
bactériocytes sont des cellules de l'hôte spécialisées dans la relation avec
son endosymbiote, dans le cas du puceron, \esp{Bushnera sp.}} du puceron,
lorsque celui-ci est soumis à un choc thermique.

\subsection{diminution antagonismes} % (fold)
\label{sub:diminution_antagonismes}

→ \esp{Wolbachia} et diminution de l'IC pendant ub HS.
\todo[inline]{Essayer de trouver le fulltext de \citet{arakaki2001}, qui parle de ça. Sinon, on vire la partie (ou on trouve une autre publi là-dessus)}