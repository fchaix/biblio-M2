\chapter{Implication du microbiote dans la réponse au stress de température} % (fold)
\label{sec:implicationµbiote}
	
	\section{Exemples clés} % (fold)
	\label{sec:exemples}
		Nous allons maintenant détailler quelques exemples de cas dans lesquels le
partenaire microbien a un rôle prépondérent dans les mécanismes de réaction au
stress thermique de leur hôte. 


		\subsection{\esp{Rickettsia sp.} et les aleurodes du tabac (\textit{whiteflies})} % (fold)
		\label{sub:rickettsia_et_les_aleurodes}
			\paragraph{description} % (fold)
\label{par:description_whitefly}

\esp{Rickettsia sp.} est un symbionte secondaire de esp{Bemisia tabaci}, un
diptère causant des maladies au tabac. Une étude \cite{brumin2011} a récement
mis en évidence un mécanisme conduisant à une meilleure résistance de la
mouche au stress thermique, lorsqu'elle est associée à sa bactérie
symbiotique.

\paragraph{Processus mis en jeu} % (fold)
\label{par:process_whitefly}

La présence de la bactérie dans les tissus de l'hôte induit une expression de
gènes associés au stress, de façon permanente, même dans des conditions
thermiques favorables. Cette présence continue de protéines de stress joue un
rôle préventif : Lors d'un stress thermique, l'insecte aura déjà exprimmé ses
gènes de résistance et résistera donc mieux au froid.

			% Via feldhaar2011

		\subsection{\esp{Buchnera sp.} et le puceron \esp{Acyrthosiphon pisum}} % (fold)
		\label{sub:exemple_buchnera}
			\paragraph{Description} % (fold)
\label{par:buchnera_description}

\cite{dunbar2007}

\paragraph{Processus mis en jeu} % (fold)
\label{par:buchnera_process}



		\subsection{Rôle de groEL dans l'hôte \esp{Nezara viridula}} % (fold)
		\label{sub:groel}
			\paragraph{Description} % (fold)
\label{par:description_groel}

% \cite{wilcox2003}
 \cite{stoll2009}

\paragraph{mécanismes mis en jeu} % (fold)
\label{par:process_groel}


	\section{Diversité des mécanismes} % (fold)
	\label{sec:diversit_des_m_canismes}
		\input{content/part2/diversité_mécanismes}

		\subsection{Stimulation de l'expression de gènes de l'hôte} % (fold)
		\label{sub:gènes_hôte}
			Exemple : partie \ref{sub:rickettsia_et_les_aleurodes} et \ref{sub:exemple_buchnera}

		\subsection{Utilisation par l'hôte de protéines microbiennes} % (fold)
		\label{sub:gènes_microbiens}
			Exemple : partie \ref{sub:groel}

		\subsection{Effet des variations de la taille des populations microbiennes sous l'effet de la température} % (fold)
		\label{sub:pop_bact}
			\input{content/part2/variations_pop_bact}
